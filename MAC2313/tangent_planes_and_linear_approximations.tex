\documentclass{article}
\usepackage{amsmath}
\usepackage{amssymb}
\usepackage[a4paper, top=0.75in, bottom=0.75in]{geometry}

\title{Tangent Planes and Linear Approximations}
\author{David Robinson}
\date{}
\setlength{\parindent}{0pt}

\begin{document}

\maketitle

\section*{Tangent Planes}

Let $P_0=(x_0,y_0,z_0)$ be a point on a surface $S$, and let $C$ be any curve passing through $P_0$ and lying entirely in $S$. If the tangent lines to all such curves $C$ at $P_0$ lie in the same plane, then this plane is called the \textbf{tangent plane} to $S$ at $P_0$.
\vspace{1em}

Let $S$ be a surface defined by a differentiable function $z=f(x,y)$, and let $P_0=(x_0,y_0)$ be a point in the domain of $f$. Then, the equation of the tangent plane to $S$ at $P_0$ is given by
\[z=f(x_0,y_0)+f_x(x_0,y_0)(x-x_0)+f_y(x_0,y_0)(y-y_0)\]

\section*{Linear Approximations}

Given a function $z=f(x,y)$ with continuous partial derivatives that exist at the point $(x_0, y_0)$, the \textbf{linear approximation} of $f$ at the point $(x_0,y_0)$ is given by the equation
\[L(x,y)=f(x_0,y_0)+f_x(x_0,y_0)(x-x_0)+f_y(x_0,y_0)(y-y_0)\]

\section*{Differentiability}

A function $f(x,y)$ is \textbf{differentiable} at a point $P(x_0,y_0)$ if, for all points $(x,y)$ in a $\delta$ disk around $P$,
\[f(x,y)=f(x_0,y_0)+f_x(x_0,y_0)(x-x_0)+f_y(x_0,y_0)(y-y_0)+E(x,y)\]
where the error term $E$ satisfies
\[\lim_{(x,y)\rightarrow(x_0,y_0)}\frac{E(x,y)}{\sqrt{{(x-x_0)}^2+{(y-y_0)}^2}}=0\]

Let $z=f(x,y)$ be a function of two variables with $(x_0,y_0)$ in the domain of $f$. If $f(x,y)$ is differentiable at $(x_0,y_0)$, then $f(x,y)$ is continuous at $(x_0,y_0)$.
\vspace{1em}

Let $z=f(x,y)$ be a function of two variables with $(x_0,y_0)$ in the domain of $f$. If $f(x,y)$, $f_x(x,y)$, and $f_y(x,y)$ all exist in a neighborhood of $(x_0,y_0)$ and are continuous at $(x_0,y_0)$, then $f(x,y)$ is differentiable there.

\section*{Differentials}

Let $z=f(x,y)$ be a function of two variables with $(x_0,y_0)$ in the domain of $f$, and let $\Delta x$ and $\Delta y$ be chosen so that $(x_0+\Delta x, y_0+\Delta y)$ is also in the domain of $f$. If $f$ is differentiable at the point $(x_0,y_0)$, then the differentials $dx$ and $dy$ are defined as $dx=\Delta x$ and $dy=\Delta y$. The differential $dz$, also called the \textbf{total differential} of $z=f(x,y)$ at $(x_0,y_0)$, is
\[dz=f_x(x_0,y_0)dx+f_y(x_0,y_0)dy\]
\end{document}