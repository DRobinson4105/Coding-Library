\documentclass{article}
\usepackage{amsmath}
\usepackage{bigints}
\usepackage{amssymb}

\title{Chapter 32 AC Circuits}
\author{David Robinson}
\date{}
\setlength{\parindent}{0pt}

\begin{document}
\maketitle

\section*{AC Sources and Phasors}

The instantaneous emf an AC oscillator can be written as $\varepsilon = \varepsilon_0\cos\omega t$
where $\varepsilon_0$ is the maximum emf and $\omega = 2\pi f$ is the angular frequency.
\vspace{1em}

The value of the emf is decreasing when $0 < \omega t < \pi$ and increasing when
$\pi < \omega t < 2\pi$.
\vspace{1em}

The resistor voltage is an AC circuit is $v_R = V_R\cos\omega t$ where $V_R$ is the maximum
voltage.
\[v_R=i_R R\]

\section*{Capacitor Circuits}
\[v_C = V_C \cos\omega t\] where $V_C$ is the maximum voltage across the capacitor.

\[q=Cv_C = CV_C \cos\omega t\]

The current is the rate of charge, $i_C = dq/dt$.
\[i_C=\frac{dq}{dt}=\frac{d}{dt}(CV_C\cos\omega t) = -\omega CV_C \sin\omega t=\omega CV_C\cos
(\omega t + \frac{\pi}{2})\]

\textbf{The AC current of a capacitor leads the capacitor voltage by $\pi / 2$ rad, or $90^\circ$.}
\vspace{1em}

Capacitive reactance, $X_C\equiv \frac{1}{\omega C}$, is represented with ohms and shows the
relationship between peak voltage and peak current.
\[I_C = \frac{V_C}{X_C}\quad\text{or}\quad V_C=I_C X_C\]

\pagebreak

\section*{RC Filter Circuits}

$V_R$ will increase steadily from $0$ to $\varepsilon_0$ as $\omega$ is increased from $0$ to very
high frequencies.

\[\omega_c = \frac{1}{RC}\]

\section*{Inductor Circuits}
\[i_L = I_L\cos(\omega t - \frac{\pi}{2})\]
where $I_L = V_L / \omega L$ and $V_L = I_L X_L$ is maximum inductor current and voltage.
\vspace{1em}

\textbf{The AC current through an inductor lags the inductor voltage by $\pi / 2$ rad, or
$90^\circ$.}

\pagebreak

\section*{The Series $RLC$ Circuit}
A circuit where a resistor, inductor, and capacitor are in series.
\begin{enumerate}
    \item The instantaneous current of all three elements is the same: $i=i_R = i_L = i_C$.
    \item The sum of the instantaneous voltages matches the emf: $\varepsilon = v_R + v_L + v_C$.
\end{enumerate}

The peak current in the $RLC$ circuit is
\[I=\frac{\varepsilon_0}{\sqrt{R^2 + {(X_L - X_C)}^2}} = \frac{\varepsilon_0}{\sqrt{R^2 + {(\omega
L - 1 / \omega C)}^2}}\]

The three peak voltags are then found from $V_R = IR$, $V_L = IX_L$, and $V_C = IX_C$.
\vspace{1em}

The denominator in the peak current equation is the impedance, $Z$, of the circuit (measured in
ohms):
\[\sqrt{R^2 + {(X_L - X_C)}^2}\]

The current is not in phase with the emf and the phase angle $\phi$ between the emf and the current
is
\[\phi = \tan^{-1}\Bigg(\frac{X_L - X_C}{R}\Bigg)\]

The resonance frequency is the frequency for the maximum current,
$I_\text{max} = \varepsilon_0 / R$, in the series $RLC$ circuit,
\[\omega_0 = \frac{1}{\sqrt{LC}}\]

\begin{itemize}
    \item At resonance, the capacitive reactance ($X_C$) and inductive reactance ($X_L$) cancel
    each other out, meaning $V_C=V_L$.
    \item If $V_C > V_L$, the circuit operates below resonance frequency.
    \item If $V_L > V_C$, the circuit operates above resonance frequency.
\end{itemize}

\section*{Power in AC Circuits}
The emf supplies energy to a circuit at the rate $p=i\varepsilon$ where $i$ and $\varepsilon$ are
the instantaneous current from and potential difference across the emf.

\subsection*{Resistors}
For a resistor,
\[p_R = i_R v_R = i_R^2 R = I_R^2 R \cos^2 \omega t\]

The average power, $P$, is the total energy dissipated per second.
\[P_R = I_R^2 R \cos^2 \omega t = I_R^2 R \Big[\frac{1}{2}(1 + \cos 2\omega t)\Big] = \frac{1}{2}
I_R^2 R + \frac{1}{2}I_R^2 R \cos 2\omega t\]

The average of $\cos 2\omega t$ is zero so
\[P_R = \frac{1}{2} I_R^2 R\]
\[P_R = {(I_\text{rms})}^2 R = \frac{{(V_\text{rms})}^2}{R} = I_\text{rms}V_\text{rms}\]
where $I_\text{rms} = \frac{I_R}{\sqrt{2}}$ is the root-mean-square current.

\subsection*{Capacitors and Inductors}
The capacitor's and inductor's average power are zero.

\subsection*{The Power Factor}
\[P_\text{source} =  \frac{1}{2} I \varepsilon_0 \cos\phi = I_\text{rms}\varepsilon_\text{rms}\cos
\phi\]
where $\cos\phi$ is the called the power factor and $\phi$ is the phase between the current and
emf.

\[P_\text{source} = P_\text{max}\cos^2 \phi\]
where $P_\text{max} = \frac{1}{2} I_\text{max} \varepsilon_0$.

\end{document}