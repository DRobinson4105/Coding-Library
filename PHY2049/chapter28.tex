\documentclass{article}
\usepackage{amsmath}
\usepackage{bigints}
\usepackage{amssymb}

\title{Chapter 28 Fundamentals of Circuits}
\author{David Robinson}
\date{}
\setlength{\parindent}{0pt}

\begin{document}
\maketitle

\section*{Kirchhoff's Laws and the Basic Circuit}
Because charge is conserved, for junctions,
\[\sum I_\text{in}=\sum I_\text{out}\]

Because energy is conserved, a charge that moves around a closed path has $\Delta U_\text{elec}=0$.
\[\Delta V_\text{loop}=\sum {(\Delta V)}_i=0\]

\section*{Energy and Power}
The rate at which the battery supplies energy to the charges is
\[P=\Delta V_R\times I=I^2 \times R=\frac{{(\Delta V_R)}^2}{R}\] where $P$ is the power in watts, $V$ is the voltage, $I$ is the current, and $R$ is the resistance.

\section*{Series Resistors}

\[R_\text{eq}=R_1 + R_2 + C_3 + \cdots\quad (\text{sequential resistors})\]
\[\Delta V_\text{eq}=\Delta V_1 + \Delta V_2 + \Delta V_3 + \cdots\quad (\text{sequential resistors})\]

\section*{Real Batteries}
Real batteries provide a slight resistance called an internal resistance.

\vspace{1em}

When a connection of very low or zero resistance is made between two points in a circuit that are normally separated by a higher resistance, a short circuit is formed which shorts out the battery. If the battery were ideal, shorting it with zero resistance would cause the current to be $I=\varepsilon / 0=\inf$. The current cannot really become infinite since the battery has an internal resistance $r$.
\[I_\text{short}=\frac{\varepsilon}{r}\]

\section*{Parallel Resistors}

\[\frac{1}{R_\text{eq}}=\frac{1}{R_1} + \frac{1}{R_2} + \frac{1}{R_3} + \cdots\quad (\text{parallel resistors})\]

\section*{Grounded Circuits}
A circuit connected to the earth is sadi to be grounded, and the wire is called the ground wire. There isn't a second wire so there is no current in the wire, it is just used as a common reference point for the potential but does not in any way change how the circuit functions.

\section*{RC Circuits}
Circuits with resistors and capacitors are called RC circuits.

If we use $\tau=RC$ as the time constant for how much the capacitor has been discharged,
\[Q=Q_0 e^{-t/\tau}\]
\[\Delta V_C=\Delta V_0 e^{-t/\tau}\]

There's no specific time at which the capacitor has been completely discharged, because $\Delta V$ approaches zero asymptotically, but the voltage and current will drop below 1\% of their initial values at some point.

\vspace{1em}

The same time constant can be used for charging a capacitor,
\[\Delta V=\varepsilon(1-e^{-t/\tau})\]
\[I=I_0 e^{-t/\tau}\quad\text((capacitor charging))\]

\end{document}