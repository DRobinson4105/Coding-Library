\documentclass{article}
\usepackage{amsmath}
\usepackage{bigints}
\usepackage{amssymb}

\title{Chapter 29 The Magnetic Field}
\author{David Robinson}
\date{}
\setlength{\parindent}{0pt}

\begin{document}
\maketitle

\section*{Magnetism}
\begin{enumerate}
    \item Magnets have two poles, and thus are magnetic dipoles
    \item Materials that are attracted to magnets are called magnetic materials
\end{enumerate}

\section*{The Discovery of the Magnetic Field}
Magnetic forces cause a compass needle to beocme aligned parallel to a magnetic field, with the
north pole of the compass showing the direction of the magnetic field at that point.

\section*{The Source of the Magnetic Field: Moving Charges}
Moving charges are the source of the magnetic field.

\[\vec{B}_\text{point charge}=\Big(\frac{\mu_0}{4\pi}\frac{qv\sin \theta}{r^2},
\text{direction given by the right-hand rule}\Big)\]
where $r$ is the distance from the charge and $\theta$ is the angle between $\vec{v}$ and
$\vec{r}$.

\vspace{1em}

The permeability constant is $\mu_0 = 1.26\times 10^{-6} \text{ T m / A}$.

\vspace{1em}

The Earth's magnetic field $5\times 10^{-5} \text{ T}$.

\subsection*{Superposition}
If there are $n$ moving point charges, the net magnetic field is given by the vector sum
\[\vec{B}_\text{total}=\vec{B}_1+\vec{B}_2 + \cdots + \vec{B}_n\]

\pagebreak

\section*{The Magnetic Field of a Current}

\subsection*{Key Magnetic Fields}
\begin{enumerate}
    \item An infinite wire: \[B=\frac{\mu_0}{2\pi}\frac{I}{r}\]
    \item A current loop: \[B_\text{center}=\frac{\mu_0}{2}\frac{NI}{R}\]
    \item A solenoid: \[B=\mu_0 nI\quad(\text{where }n=N/L)\]
\end{enumerate}

\[\vec{B}_\text{current segment}=\frac{\mu_0}{4\pi}\frac{I\Delta \vec{s}\times \hat{r}}{r^2}
\quad\text{(magnetic field of a very short segment of current)}\]

\subsection*{Key Points}
\begin{enumerate}
    \item A circular loop of wire with a circulating current is called a current loop.
\end{enumerate}

\section*{Magnetic Dipoles}
The on-axis magnetic field of a current loop is
\[B_\text{loop}=\frac{\mu_0}{2}\frac{IR^2}{{(z^2+R^2)}^{3/2}}=\frac{\mu_0}{2\pi}\frac{(\pi R^2)I}
{{(z^2+R^2)}^{3/2}}=\frac{\mu_0}{2\pi}\frac{AI}{{(z^2+R^2)}^{3/2}}\]
where $A=\pi R^2$ is the area of the loop. 

\vspace{1em}

The magnetic dipole $\vec{m}$ of a current loop enclosing area $A$ is 
\[\vec{m}=(AI\text{, from the south pole to the north pole})\: A\: m^2\]

\[\vec{B}_\text{dipole}=\frac{\mu_0}{2\pi}\frac{\vec{m}}{z^3}\quad
\text{(on the axis of a magnetic dipole)}\]

\pagebreak

\section*{Ampere's Law and Solenoids}
\subsection*{Line Integrals}
\begin{enumerate}
    \item If the magnetic field is everywhere perpendicular to a line, then \[\int_i^f\vec{B}
    \cdot d\vec{s}=0\]
    \item If the magnetic field is everywhere tangent to a line, then \[\int_i^f\vec{B}\cdot d
    \vec{s}=Bl\]
\end{enumerate}

\subsection*{Ampere's Law}
Since the magnetic field strength of a current-carrying wire is $B=\mu_0 I / 2\pi r$, the line
integral for a wire in a circle can be simplified to
\[\oint \vec{B}\cdot d \vec{s} = Bl = \frac{\mu_0 I}{2\pi r}\times 2\pi r=\mu_0 I\]

\subsection*{The Magnetic Field of a Solenoid}
The magnetic field of a solenoid is uniform at every point in the solenoid and is parallel to the
axis, whereas the field outside the loops is very close to zero.

\pagebreak
\section*{The Magnetic Force on a Moving Charge}
\subsection*{Magnetic Force}
A magnetic field exerts a force on a moving charge and is perpendicular to the direction of the
charge's velocity and the magnetic field. Since the magnetic foerce is always perpendicular to the
particle's displacement, it does no work.

\[\vec{F}_\text{on q}=q\vec{v}\times \vec{B}=(qvB\sin\alpha\text{, direction of right-hand rule})\]
where $\alpha$ is the angle between $\vec{v}$ and $\vec{B}$.

\subsection*{Cyclotron Motion}
Since the force for circular motion is $F=qvB=ma_r=\frac{mv^2}{r}$, the radius of the cyclotron
orbit is $r_\text{cyc}=\frac{mv}{qB}$.
\vspace{1em}

Since the frequency for cicular motion is $f=v/2\pi r$, the cyclotron freqency is
\[f_\text{cyc}=\frac{qB}{2\pi m}\]

\subsection*{The Hall Effect}
For a flat, current-carrying conductor, the magnetic field would be perpendicular to the top
surface of it. As positive charges and negative charges pass through it, the magnetic force will
cause a buildup of positive charges on one side and negative on the other. This causes the sides to
act like the charge on the plates of a capacitor, creating a potential difference $\Delta V$
between the two surfaces and an electric field $E=\Delta V/\omega$ between them.
\vspace{1em}

This potential differnce is called the Hall voltage
\[\Delta V_H=\omega v_d B\]
$v_d$ is the drift speed of the charge-carriers, $v_d=\frac{I}{\omega t n e}$ where $A=\omega t$ is
the cross-section area of the conductor.
\[\Delta V_H=\frac{IB}{tne}\]

\section*{Magnetic Forces on Current-Carrying Wires}
\[\vec{F}_\text{wire}=I\vec{l}\times \vec{B}=(IlB\sin\alpha\text{, direction of right-hand rule})\]
\[F_\text{parallel wires}=I_1 l B_2 = I_1 l \frac{\mu_0 I_2}{2\pi d}=\frac{\mu_0 l \: I_1 I_2}
{2\pi d}\]
Parallel wires carrying currents in the same direction attract each others, while parallel wires
carrying currents in opposite directions repel each other.

\section*{Forces and Torques on Current Loops}
Magnetic poles attract or repel because the moving charges in one current exert attractive or
repulsive magnetic forces on the moving charges in the other current.
\[\vec{\tau}=\vec{m}\times\vec{B}\]

\section*{Magnetic Properties of Matter}
\begin{enumerate}
    \item Electrons are microscopic magnets due to their spin.
    \item A ferromagnetic (iron-like) material in which the spins are aligned is organized into
    magnetic domains.
    \item The individual domains align with an external magnetic field to produce an induced
    magnetic dipole moment for the entire object.
\end{enumerate}
\end{document}