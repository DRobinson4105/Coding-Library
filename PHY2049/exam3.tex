\documentclass[twocolumn]{article}
\usepackage{amsmath}
\usepackage[a4paper, top=0.75in, bottom=0.75in, left=0.75in, right=0.75in]{geometry}

\setlength{\parindent}{0pt}

\begin{document}
\subsubsection*{Moving Conductor}
\[\Delta V = \varepsilon = vlB\]
\[I = \frac{\varepsilon}{R} = \frac{vlB}{R}\]
\[F_\text{mag}=F_\text{pull}=IlB=\frac{vl^2 B^2}{R}\]
\[P_\text{input}=P_\text{dissipated}=I^2 R=\frac{v^2 l^2 B^2}{R}\]

\[\Phi_m=\int\vec{B}\cdot d\vec{A}\]
\[\Phi_m = \vec{A}\cdot\vec{B}=|A||B|\cos\theta\quad\text{(uniform magnetic field)}\]

\begin{itemize}
    \item Increasing flux: The induced magnetic field points opposite the applied magnetic
    field.
    \item Decreasing flux: The induced magnetic field points in the same direction as the
    applied magnetic field.
    \item Steady flux: There is no induced magnetic field.
\end{itemize}

\[\varepsilon_\text{induced} = \frac{d\Phi_m}{dt}\]
\[I_\text{induced}=\frac{\varepsilon_\text{induced}}{R}\]
\[E_\text{inside}=\frac{r}{2}\Big|\frac{dB}{dt}\Big|\quad\text{Solenoid}\]
\[\frac{V_2}{V_1} = \frac{N_2}{N_1}\quad\text{Transformers}\]
\subsubsection*{Inductors}
\[L=\frac{\Phi_m}{I}\text{ henry (H)}\quad\text{Inductance}\]
\[\Delta V_L = -L\frac{dI}{dt}\]
\[U_L = L\int_{0}^{I}IdI = \frac{1}{2}LI^2\]

\subsubsection*{LC Circuits}
\[I=-\frac{dQ}{dt}\]
\[Q(t)=Q_0\cos\omega t\]
\[\omega = \frac{1}{\sqrt{LC}}\quad f =\omega / 2\pi \]

\subsubsection*{LR Circuits}
\[I=I_0 e^{-t/(L / R)}\]
\[\tau = \frac{L}{R}\quad\text{where current has decreased to }e^{-1}\]
\subsubsection*{Right-hand rule (wire)}
\begin{enumerate}
    \item Point thumb in the direction of current
    \item Point fingers in the direction of magnetic field
    \item Point palm in the face of force on wire
\end{enumerate}


\subsubsection*{Right-hand rule (electromagnetic waves)}
\begin{enumerate}
    \item Point index finger in the direction of electric field
    \item Point middle finger in the direction of magnetic field
    \item Point thumb in the direction of motion
\end{enumerate}

\[X_C = \frac{1}{2\pi fC}\] where $X_C$ is the capacitive reactance in ohms, $f$ is the frequency,
and $C$ is the capacitance.

\end{document}