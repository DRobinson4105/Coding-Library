% chktex-file 44

\documentclass[8pt,twocolumn]{extarticle}
\usepackage{amsmath}
\usepackage[a4paper, top=0.25in, bottom=0.25in, left=0.25in, right=0.25in]{geometry}

\setlength{\parindent}{0pt}

\begin{document}
Permittivity of free space: $\epsilon_0 = 8.854\times 10^{-12}$ |
Proton mass: $1.673\cdot 10^{-27}$

Elementary charge: $e = 1.602\cdot 10^{-19}$ |
Electron mass: $9.11\cdot 10^{-31}$

Speed of light: $c=3\cdot 10^8$ |
Permeability constant: $\mu_0 = 1.26\times 10^{-6}$

$k=9\times 10^9$
\[\lambda = \frac{Q}{L}\quad\eta = \frac{Q}{A}\quad\rho = \frac{Q}{V}\]
\[E=\frac{kq}{r^2}\quad F=qE=\frac{kq_1 q_2}{r^2}\quad V=\frac{kq}{r}\quad U=qV=\frac{kq_1 q_2}{r}\quad\text{\textbf{(point)}}\]
\[E_\text{wire}=\frac{2k|\lambda |}{r}=\frac{2kQ r^2}{R^3 L}\quad E_\text{plane}=\frac{\eta}{2\epsilon_0}\quad E_\text{ring}=\frac{kzQ}{{(z^2 + r^2)}^{3/2}}\]
\[E_\text{disk}=\frac{\eta}{2\epsilon_0}\Big(1-\frac{z}{\sqrt{z^2 + R^2}}\Big)\quad E_\text{sphere}=\begin{cases}
    \frac{kQ}{r^2} & \text{for } r >= R \\ \\
    \frac{kQr}{R^3} & \text{for } r < R
\end{cases}\]
\[\text{Dipole}\begin{cases}
    p=qd\quad E=-\frac{2kp}{r^3}\text{(on axis)}\quad E=-\frac{kp}{r^3}\text{\textbf{(on bisecting plane)}} \\
    \tau = pE\sin\theta\quad V = -pE\cos\theta
\end{cases}\]
\[\text{Parallel-Plate Capacitor}\begin{cases}
    E=\frac{V}{d}=\frac{Q}{\epsilon_0 A}\text{\textbf{(positive to negative plate)}} \\
    s=\text{distance from negative plate} \\
    U=qEs=qV \quad V=\frac{s}{d}\Delta V_C
\end{cases}\]
\[\Phi_\text{enc}=EA=\frac{Q_\text{in}}{\epsilon_0}\quad E_\text{conductor surface}=\frac{\eta}{\epsilon_0}\]
\[W=-q\Delta V\quad \Delta V = -Ed\quad V_\text{sphere} = \frac{4}{3}\pi r^3\quad A_\text{sphere}=4\pi r^2\]
\[W=Q\times V\quad C = \frac{Q}{\Delta V_C}\]
\[C=\frac{\epsilon_0 A}{d}\quad Q = \frac{\epsilon_0 A}{d} \Delta V_C\quad\text{\textbf{(parallel-plate capacitor)}}\]
\[C_\text{eq}=\begin{cases}
    C_1 + C_2 + \cdots & \text{\textbf{(parallel)}} \\
    {(\frac{1}{C_1} + \frac{1}{C_2} + \cdots)}^{-1} & \text{\textbf{(sequential)}}
\end{cases}\]
\[C=\kappa C_0\]
\[i_e = n_e Av_d\quad I = n\times A\times e\times v_d\quad n=\text{electron density}\]
\[\text{Current density}=J=\frac{I}{A}=n_e e v_d\quad I = JA = \sigma AE\quad \sigma = \text{conductivity}\]
\[\rho = \frac{1}{\sigma} = \text{resistivity}\quad R=\rho \frac{L}{A}\quad I=\frac{A}{\rho L}\Delta V\quad \Delta V = IR\]
\[R_\text{eq}=\begin{cases}
    {(\frac{1}{R_1} + \frac{1}{R_2} + \cdots)}^{-1} & \text{\textbf{(parallel)}} \\
    R_1 + R_2 + \cdots & \text{\textbf{(sequential)}}
\end{cases}\]
If one branch in a parallel circuit is opened, the current through the other stays the same
\vspace{1em}

All parts of a sequential wire have the same current
\[P=\Delta V_R\times I = I^2 \times R = \frac{{(\Delta V_R)}^2}{R}\]
\[\tau = RC\quad Q=Q_0 e^{-t/\tau}\quad \Delta V_C = \Delta V_0 e^{-t/\tau}\]
Right-hand rule for wire (wrap fingers around wire): thumb points toward current and fingers point toward magnetic field
\vspace{1em}

Right-hand rule for magnetic field: thumb points toward force, palm faces magnetic field, fingers point toward motion
\[\oint B\cdot dl=Bl=\frac{\mu_0 I}{2\pi r}\times 2 \pi r=\mu_0 I\quad \Phi_m = BA\cos\theta\]
\[F_B = qv\times B = IL\times B\]
\[B_\text{point charge} = \frac{\mu_0 q (\vec{v}\times\vec{r})}{4\pi r^3} = \frac{\mu_0 qv\sin\theta}{4\pi r^2}\quad B_\text{wire} = \begin{cases}
    \frac{\mu_0 I}{2\pi r} & r >= R \\
    \frac{\mu_0 I r}{2\pi R^2} & r < R
\end{cases}\]
\[B_\text{center of current loop}=\frac{\mu_0 N I}{2R}\quad B_\text{solenoid}=\mu_0 n I\quad n=N/L\]
\[\vec{B}_\text{current segment}=\frac{\mu_0 I \Delta \vec{s}\times \hat{r}}{4\pi r^2}\]
\[\text{Magnetic dipole moment}=\vec{m}=(AI, \text{from south to north pole})\]
\[\vec{B}_\text{on axis of dipole}=\frac{\mu_0 \vec{m}}{2\pi z^3}\]
\[A\times B=(A_y B_z - A_z B_y)\mathbf{i} + (A_z B_x - A_x B_z)\mathbf{j} + (A_x B_y - A_y B_x)\mathbf{k}\]

\begin{center}
    \begin{tabular}{|c c c c|} 
        \hline
        Metal & Electron density ($m^{-3}$) & Resistivity & Conductivity \\ [0.5ex]
        \hline
        Aluminum & $18\times 10^{28}$ & $2.8\times 10^{-8}$ & $3.5\times 10^7$ \\
        \hline
        Iron & $17\times 10^{28}$ & $9.7\times 10^{-8}$ & $1.0\times 10^7$ \\
        \hline
        Copper & $8.5\times 10^{28}$ & $1.7 \times 10^{-8}$ & $6.0\times 10^7$ \\
        \hline
        Gold & $5.9\times 10^{28}$ & $2.4\times 10^{-8}$ & $4.1\times 10^7$ \\
        \hline
        Silver & $5.8\times 10^{28}$ & $1.6\times 10^{-8}$ & $6.2\times 10^7$ \\
        \hline
    \end{tabular}
\end{center}

Kirchhoff's Law: Branch out from one terminal and set up equations for $\sum \Delta V_i = 0$
\vspace{1em}

There is no current in the ground wire

\[\Delta V = \varepsilon = vlB\quad\text{\textbf{(moving conductor)}}\]
\pagebreak

Increasing flux: induced magnetic field points opposite to applied

Decreasing flux: induced magnetic field points same direction to applied

Steady flux: no induced magnetic field
\[\varepsilon_\text{induced} = -\frac{d\Phi_m}{dt}\quad I_\text{induced}=\frac{\varepsilon_\text{induced}}{R}\]
\[E_\text{inside}=\frac{r}{2}\Big|\frac{dB}{dt}\Big|\quad\text{solenoid}\]
\[\frac{V_2}{V_1} = \frac{N_2}{N_1}\quad P_1=P_2\quad V_1 I_1 = V_2 I_2\quad\frac{I_1}{I_2}=\frac{N_2}{N_1}\quad\text{\textbf{(transformers)}}\]
\[L=\frac{\Phi_m}{I}\quad \Delta V_L = -L\frac{dI}{dt}\quad U_L = L\int_{0}^{I}IdI = \frac{1}{2}LI^2\quad\text{\textbf{(inductors)}}\]
\[I=-\frac{dQ}{dt}\quad Q(t)=Q_0\cos\omega t\quad \omega = \frac{1}{\sqrt{LC}}\quad\text{\textbf{(LC circuits)}}\]
\[I=I_0 e^{-t/\tau}\quad\tau = \frac{L}{R}\quad\text{\textbf{(LR circuits)}}\]
\[\vec{E}_B = \vec{E}_A + \vec{v}_{BA}\times \vec{B}_A\quad \vec{B}_B = \vec{B}_A - \frac{1}{c^2}\vec{v}_{BA}\times \vec{E}_A\]
Displacement current is from changing electric field rather than flow of charges.
\[I_\text{disp}=\epsilon_0\frac{d\Phi_e}{dt}=\epsilon_0 \cdot A \frac{dE}{dt}\]
\[\oint\vec{E}\cdot d\vec{A}=\frac{Q_\text{in}}{\epsilon_0}\quad\text{(Gauss's Law)}\]
\[\oint\vec{B}\cdot d\vec{A}=0\quad\text{(Gauss's Law for Magnetism)}\]
\[\oint\vec{E}\cdot d\vec{s} = -N\frac{d\Phi_m}{dt}\quad\text{(Faraday's Law)}\]
\[\oint\vec{B}\cdot d\vec{s} = \mu_0 I_\text{through} + \epsilon_0 \mu_0 \frac{d\Phi_e}{dt}\quad\text{(Ampere-Maxwell Law)}\]
\[\vec{F}=q(\vec{E} + \vec{v} \times \vec{B})\quad\text{(Lorentz Force Law)}\]
\[E(x, t) = E_0 \cos\left(kx - \omega t + \phi\right)\quad c=\frac{\omega}{k}=\frac{\omega\lambda}{2\pi}=f\lambda=\frac{E}{B}\]
\[I=\frac{P}{4\pi r^2}=\frac{c\epsilon_0 E_0^2}{2}\]
Average value for $\sin\theta$ is $\frac{1}{2}$
\[\langle S \rangle = \frac{1}{\mu_0}EB\sin\theta\quad\text{(\textbf{Poynting Vector} --- energy flux of an EM wave)}\]
Right-hand rule for electromagnetic waves: fingers point toward electric field, palm faces magnetic field, thumb points toward motion
\[I_\text{transmitted}=\frac{1}{2} I_0\quad\text{\textbf{(unpolarized)}}\quad I_\text{transmitted}=I_0\cos^2
\theta\quad\text{\textbf{(polarized)}}\]
\[\varepsilon=\varepsilon_0 \cos\omega t\quad \omega=2\pi f\quad X=\text{Reactance}\]
\[v_R = i_R R = V_\text{max}\sin\omega t\quad p=i\varepsilon\quad\text{\textbf{(AC circuit)}}\]
\[\text{Capacitor circuit}\begin{cases}
    v_C = V_C\cos\omega t\quad q=Cv_C\quad X_C=\frac{1}{\omega C}\quad I_C=\frac{V_C}{X_C}\quad \\
    i_C=-\omega CV_C\sin\omega t=\omega CV_C\cos(\omega t + \frac{\pi}{2})
\end{cases}\]
\[\omega_C = \frac{1}{RC}\quad\text{\textbf{(RC Circuit)}}\]
An inductor is a coil of wire that generates a magnetic field when current flows through it and
resist changes in current by inducing an emf opposite to the charge.
\[\text{Inductor circuit}\begin{cases}
    i_L = I_L\cos(\omega t - \frac{\pi}{2})\quad V=L\cdot \frac{dI}{dt} \\ \\
    X_L=\omega L\quad I_L=\frac{V_L}{X_L}
\end{cases}\]
\[\text{Series RLC Circuit}\begin{cases}
    Z=\sqrt{R^2 + {(X_L - X_C)}^2}\quad\text{\textbf{(impedance)}} \\
    V=\sqrt{V_R^2 + {(V_L - V_C)}^2}\quad I_\text{peak}=\frac{\varepsilon_0}{Z} \\
    \phi_\text{between emf and current} = \tan^{-1}\Big(\frac{X_L - X_C}{R}\Big) \\
    \omega_0 = \frac{1}{\sqrt{LC}}=\text{\textbf{resonance angular frequency}}
\end{cases}\]
Resonance frequency occurs when $X_L=X_C$ and $Z=R$. If $V_C > V_L$, the circuit operates below resonance frequency. If $V_L > V_C$, the circuit operates above resonance frequency.
\[P_R = \frac{1}{2}I_R^2 R=I_\text{rms}V_\text{rms}\quad x_\text{rms}=\frac{x}{\sqrt{2}}\]
\[P_\text{source}=\frac{1}{2}I\varepsilon_0\cos\phi = I_\text{rms}\varepsilon_\text{rms}\cos\phi=
P_\text{max}\cos^2\phi\]
where $\cos\phi$ is the power factor, $\phi$ is the phase between current and emf, and $P_\text{max}=
\frac{1}{2}I_\text{max}\varepsilon_0$.
\begin{itemize}
    \item AC circuit with capacitor: current leads voltage by $\frac{\pi}{2}$ (current reaches
    maximum $\frac{T}{4}$ before voltage)
    \item AC circuit with inductor: current lags voltage by $\frac{\pi}{2}$ (current reaches
    maximum $\frac{T}{4}$ after voltage)
    \item AC circuit with resistor: current is in phase with voltage
\end{itemize}
\end{document}