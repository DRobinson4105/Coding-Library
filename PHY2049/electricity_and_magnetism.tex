\documentclass{article}
\usepackage{amsmath}

\begin{document}
\setlength{\parindent}{0pt}

\section*{Electricity and Magnetism}

Two fundamental properties of matter determine interactions between objects in classical physics
\textbf{mass} and \textbf{electric charge}.
\begin{enumerate}
    \item \textbf{Gravitational force} attracts objects together and is proportional to masses of
    the objects.
    \[\mathbf{\vec{F}_g} = G \frac{Mm}{r_{12}^2} \hat{r}\]
    \item \textbf{Electromagnetic interaction} causes like charges to repel while opposite charges
    attract to each other, keeps electrons bound to nuclei, and causes chemical binding to keep
    atoms together in molecules or solid materials.
    
\end{enumerate}
\section*{Electric Charge}

\begin{itemize}
    \item Electric charge can't be created from nothing and can't be destroyed.
    \item The process of removing an electron from the electron cloud of an atom, or adding an
    electron to it, is called \textbf{ionization}.
    \item The electrons in an \textbf{insulator} (glass, ceramic, plastic) are all tightly bound to
    the positive nuclei and do not move around so they are immobile.
    \item The outer electrons in a \textbf{conductor} (copper, silver, aluminum) are only weakly
    bound to the nuclei so they can easily become detached.
\end{itemize}
\begin{center}
\begin{tabular}{|c c c c|} 
    \hline
    Type & Charge & Mass & Mass in SI \\ [0.5ex] 
    \hline
    electron & $-e$ & $1$ & $9.11 * 10^{-31}$ kg \\ 
    \hline
    proton & $+e$ & $1836$ & $1.673 * 10^{-27}$ kg \\
    \hline
    neutron & $0$ & $1839$ & $1.675 * 10^{-27}$ kg \\
    \hline
    positron & $+e$ & $1$ & $9.11 * 10^{-31}$ kg \\
    \hline
\end{tabular}
\end{center}

One elementary charge, $e$ is equal to $1.602*10^{-19} \: C$.

\section*{Electrostatics}
\begin{itemize}
    \item If a charge is moving, it creates a magnetic field that exerts forces on other moving charges.
    \item If a charge is moving with acceleration, it emits electro-magnetic waves which also exert forces on other charges.
\end{itemize}
If charges move slow with small acceleration the effects of magnetic field and electromagnetic
waves can be neglected.
\subsection*{Coulomb's Law}
\[\pmb{F_{12}}=k\frac{q_1 q_2}{r_{12}^2}\pmb{\hat{r}_{12}}\]
where: \\
$k={(4\pi\varepsilon_0)}^{-1}=9.0*10^9 \: N m^2 / C^2$ \\
$\varepsilon_0=$ permittivity of free space $= 8.86 * 10^{-12} \: C^2 / N m^2$

\section*{Electric Field}
Electric field is the force exerted by source charges on a positive unit test charge
\[\pmb{E(r)}=k\frac{Q}{r^2}\pmb{\hat{r}}\]
If the source charge $Q$ is positive, the direction of electric field vector is away from $Q$,
otherwise the electric field vector is directed toward $Q$. Direction of $E$ is the same as
direction of force acting on positive charge. \\ \\
If we know strength of the electric field, $E$, and the test charge value $q_t$, we can obtain the
force acting on $q_t$ with the formula, $\pmb{F} = q_t \pmb{E(r)}$

\end{document}