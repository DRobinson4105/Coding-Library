\documentclass{article}
\usepackage{amsmath}
\usepackage{bigints}

\title{Chapter 25 The Electric Potential}
\author{David Robinson}
\date{}
\setlength{\parindent}{0pt}

\begin{document}
\maketitle

\section*{Electric Potential Energy}
A charged particle $q$ in an electric potential $V$ has electric potential energy \[U=qV\]

The change in potential energy is $\Delta U=-W$ where $W$ is the work being done. If the force is
not constant, the work can be calculated by dividing the path into many small segments of length
$dx$, finding the work done in each segment, and then integrating over it.
\[W=\int_{x_i}^{x_f}F(x)\cos\theta dx\]

\subsubsection*{Parallel-Plate Capacitor}
The electric potential energy of charge $q$ in a parallel-plate capacitor is
\[U_\text{elec}=qEs\] where $s$ is the distance from the negative plate.

\subsubsection*{Two Point Charges}
The potential energy of the system of two charges $q_1$ and $q_2$ is
\[U_\text{elec}=\frac{kq_1 q_2}{r}\]

\subsubsection*{Multiple Point Charges}
If more than two charges are present, their potential energy is the sum of the potential energies
due to all pairs of charges.

\subsubsection*{Electric Dipole in a Uniform Electric Field}
\[U_\text{elec}=pE\cos\phi=\vec{p}\cdot \vec{E}\]

\section*{Electric Potential}
In the absence of other appleid forces, a charged particle speeds up or slows down as it moves
through a potential difference. 

\subsubsection*{Parallel-Plate Capacitor}
The electric potential inside a capacitor is
\[V_\text{cap}=\frac{s}{d}\Delta V_C\]
where $s$ is the distance from the negative plate, $d$ is the distance between the plates, and
$\Delta V_C$ is the potential difference between the plates.

\subsection*{Point Charge}
The electric potential of charge $q$ is
\[V_\text{point}=\frac{kq}{r}\]

\subsection*{Charged Sphere}
Outside a uniformly charged sphere, the electric potential is identical to that of a point charge
$Q$ at the center
\[V_\text{sphere}=\frac{kQ}{r}\]

A sphere is charged to a certain potential when that potential is on the surface of the sphere,
such that a sphere of radius $R$ that is charged to potential $V_0$ has total charge
\[Q=\frac{RV_0}{k}\]

Using this for $Q$ in the electric potential equation results In
\[V_\text{sphere}=\frac{R}{r}V_0\]
\end{document}