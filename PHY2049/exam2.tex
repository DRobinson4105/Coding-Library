\documentclass[twocolumn]{article}
\usepackage{amsmath}
\usepackage[a4paper, top=0.75in, bottom=0.75in, left=0.75in, right=0.75in]{geometry}

\setlength{\parindent}{0pt}

\begin{document}

Permeability constant: $\mu_0 = 1.26\times 10^{-6}$

Permittivity of free space: $\epsilon_0 = 8.854\times 10^{-12}$

Elementary charge: $e = 1.602\cdot 10^{-19}$

Proton mass: $1.673\cdot 10^{-27}$

Electron mass: $9.11\cdot 10^{-31}$
\[\Delta V=-E_s\Delta s\]
\[W=Q\times V\]
\[C=\frac{Q}{\Delta V_C}= \text{ with units F or farad}\]
\[C=\frac{\epsilon_0 A}{d}\quad\text{(parallel-plate capacitor)}\]
\[Q=\frac{\epsilon_0 A}{d}\Delta V_C\quad\text{(parallel-plate capacitor)}\]
\[C_\text{eq}=C_1 + C_2 + C_3 + \cdots\quad (\text{parallel capacitors})\]
\[\frac{1}{C_\text{eq}}=\frac{1}{C_1} + \frac{1}{C_2} + \frac{1}{C_3} + \cdots\quad
(\text{sequential capacitors})\]
\[C=\kappa C_0\]
Wires in series have the same current
\[i_e = n_e Av_d\]
\[I=n\times A\times e\times v_d\] where n is electron density and e is elementary charge
\begin{center}
    \begin{tabular}{|c c|} 
        \hline
        Metal & Electron density ($m^{-3}$) \\ [0.5ex]
        \hline
        Aluminum & $18\times 10^{28}$ \\
        \hline
        Iron & $17\times 10^{28}$ \\
        \hline
        Copper & $8.5\times 10^{28}$ \\
        \hline
        Gold & $5.9\times 10^{28}$ \\
        \hline
        Silver & $5.8\times 10^{28}$ \\
        \hline
    \end{tabular}
\end{center}
\[I=\frac{dQ}{dt} \text{ with units A or ampere or C/s}\]
\[\text{Current density: } J=\frac{I}{A}=n_e e v_d\]
\[I=JA=\sigma AE\]
\begin{center}
    \begin{tabular}{|c c c|} 
        \hline
        Metal & Resistivity & Conductivity \\ [0.5ex]
        \hline
        Aluminum & $2.8\times 10^{-8}$ & $3.5\times 10^7$ \\
        \hline
        Iron & $9.7\times 10^{-8}$ & $1.0\times 10^7$ \\
        \hline
        Copper & $1.7 \times 10^{-8}$ & $6.0\times 10^7$ \\
        \hline
        Gold & $2.4\times 10^{-8}$ & $4.1\times 10^7$ \\
        \hline
        Silver & $1.6\times 10^{-8}$ & $6.2\times 10^7$ \\
        \hline
    \end{tabular}
\end{center}
Resistivity, $\rho = 1/\sigma$, is the inverse of the conductivity.
\[R=\rho \frac{L}{A}\]
\[I=\frac{A}{\rho L}\Delta V\]
\[\Delta V = IR\]
\[R_\text{eq}=R_1 + R_2 + R_3 + \cdots\quad (\text{sequential resistors})\]
\[\frac{1}{R_\text{eq}}=\frac{1}{R_1} + \frac{1}{R_2} + \frac{1}{R_3} + \cdots\quad
(\text{parallel resistors})\]
If one branch in a parallel circuit is opened, the current through the other stays the same
\[P=\Delta V_R\times I = I^2\times R = \frac{{(\Delta V_R)}^2}{R}\]
\[\tau = RC\]
\[Q=Q_0 e^{-t/\tau}\]
\[\Delta V_C=\Delta V_0 e^{-t/\tau}\]
Right-hand rule for wire
\begin{enumerate}
    \item Thumb is in direction of current
    \item If from wire, fingers are curled around the wire
    \item Fingers point in the direction of magnetic field
\end{enumerate}
Right-hand rule for magnetic field
\begin{enumerate}
    \item Thumb is in direction of force
    \item Palm is facing the magnetic field
    \item Fingers point in the direction of motion
\end{enumerate}
\[\oint B\cdot dl = Bl = \frac{mu_0 I}{2\pi r}\times 2\pi r = \mu_0 I_\text{enc}\]
\[\Phi_b = BA\cos\theta\]
\[F_B=qv\times B=IL\times B\]
\[\vec{B}_\text{point charge}=\Big(\frac{\mu_0}{4\pi}\frac{qv\sin \theta}{r^2},
\text{direction given by the right-hand rule}\Big)\]
\begin{enumerate}
    \item An infinite wire: $B=\frac{\mu_0}{2\pi}\frac{I}{r}$ if \textbf{outside} of the wire and
    $B=\frac{\mu_0}{2\pi}\frac{Ir}{R^2}$ if \textbf{inside} the wire where $r$ is the distance from
    the center and $R$ is the radius of the wire
    \item A current loop: $B_\text{center}=\frac{\mu_0}{2}\frac{NI}{R}$
    \item A solenoid: $B=\mu_0 nI\quad(\text{where }n=N/L)$
\end{enumerate}

\[\vec{B}_\text{current segment}=\frac{\mu_0}{4\pi}\frac{I\Delta \vec{s}\times \hat{r}}{r^2}\]
\quad(magnetic field of a very short segment of current)
\vspace{1em}

Magnetic dipole moment $\vec{m}=$ ($AI$, from the south pole to the north pole)
\[\vec{B}_\text{dipole}=\frac{\mu_0}{2\pi}\frac{\vec{m}}{z^3}\quad
\text{(on the axis of a magnetic dipole)}\]
\[A\times B = (A_yB_z - A_zB_y)\mathbf{i}+(A_zB_x-A_xB_z)\mathbf{j}+(A_xB_y-A_yB_x)\mathbf{k}\]

\end{document}