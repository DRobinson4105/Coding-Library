\documentclass{article}
\usepackage{amsmath}
\usepackage{bigints}

\begin{document}
\setlength{\parindent}{0pt}

\section*{Electric Flux}
Electric flux, $\Phi$, of the electric field, $E$, through the surface, $A$, as:
\[\Phi=E\cdot A=EA\cos\theta\]

If the surface is curved, then we can
\begin{enumerate}
    \item Divide the surface into small regions with area $dA$.
    \item $d\Phi=E\cdot dA$
    \item Obtain the total flux by integrating over the surface $A$, $\Phi=\bigintsss E\cdot dA$
\end{enumerate}

\subsection*{Electric Field in a Sphere}
If there is a positive point charge $q$, through a spehere of radius $R$ centered at the charge, the electric field is $E=\frac{kq}{r^2}$ at any distance $r >= R$. \\

In a solid sphere of radius $R$ with the uniformly distributed charge $Q$, find $E(r)$ at a point inside the sphere by:
\begin{itemize}
    \item $Q_{enc} = \frac{r^3}{R^3}Q$
    \item So for $r < R$, $E(r)=\frac{kQ \times r^3 / R^3}{r^2}=\frac{kQr}{R^3}$
\end{itemize}

\subsection*{Gauss's Law}
Electric flux through any closed surface always equals:
\[\bigintsss E \cdot dA = \frac{Q_{enc}}{\epsilon_0}\] where $Q_{enc}$ is the enclosed charge

% \begin{center}
%     \begin{tabular}{|c | c | c | c|} 
%         \hline
%          & Sphere (or point) & Cylindrical (or line) & Planar \\ [0.5ex] 
%         \hline
%         Charge Density & Depends only on radial distance from central point & Depends only on perpendicular distance from line & Depends only on perpendicular distance from plane \\ 
%         \hline
%         Gaussian Surface & Centered at point of symmetry & Centered at axis of symmetry & Cylinder with axis perpendicular to plane \\
%         \hline
%         Electric Field & $E$ constant at surface $E \parallel A - \cos\theta = 1$ & $E$ constant at curved surface and \\
%         \hline
%     \end{tabular}
% \end{center}

\subsection*{Electric Flux of Multiple Charges}
In the case of multiple charges through any closed surface:
\begin{itemize}
    \item The contribution to the total flux for any charge $q_i$ inside the surface is $q_i/\epsilon_0$.
    \item The contribution to the total flux for any charge outside the surface is zero.
    \item $\Phi = Q_{in} / \epsilon_0$ where $Q_{in}$ is the sum of the charges inside the surface
\end{itemize}

\end{document}