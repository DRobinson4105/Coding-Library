\documentclass{article}
\usepackage{amsmath}
\usepackage{bigints}
\usepackage{amssymb}

\title{Chapter 27 Current and Resistance}
\author{David Robinson}
\date{}
\setlength{\parindent}{0pt}

\begin{document}
\maketitle

\section*{The Electron Current}
Current is the flow of charge through a conductor.

\subsection*{Charge Carriers}
The drift speed is the net motion of the individual electrons through a conductor, which is
typically $v_d = 10^{-4} m/s$. The number of electrons $N_e$ of electrons that pass through the
cross section during the time interval $\Delta t$ is
\[N_e=i_e \Delta t\] where $i_e$ is the electron current which is the number of electrons per
second that passes through a cross section of a conductor.

\vspace{1em}

The electrons travel distance $\Delta x = v_d \Delta t$, forming a cylinder of charge with volume
$V = A\Delta x$. If the electron density is $n_e$ electrons per cubic meter, then the total number
of electrons in the cylinder is
\[N_e = n_e V = n_e A \Delta x = n_e A v_d \Delta t\] resulting in \[i_e = n_e Av_d\]

\subsubsection*{Electron density in metals}

\begin{center}
\begin{tabular}{|c c|} 
    \hline
    Metal & Electron density ($m^{-3}$) \\ [0.5ex]
    \hline
    Aluminum & $18\times 10^{28}$ \\
    \hline
    Iron & $17\times 10^{28}$ \\
    \hline
    Copper & $8.5\times 10^{28}$ \\
    \hline
    Gold & $5.9\times 10^{28}$ \\
    \hline
    Silver & $5.8\times 10^{28}$ \\
    \hline
\end{tabular}
\end{center}

\pagebreak

\section*{Creating a Current}
An electron current is a net motion of charges sustained by an internal electric field.
\vspace{1em}

At the moment a positive charged wire and a negative charged wire are connected, the surface charge
density varies from positive at the positive capacitor plate through zero at the midpoint to
negative at the negative plate. The varying surface charge distribution creates an internal
electric field inside the wire, causing a current.
\[i_e=\frac{n_e e \tau A}{m} E\] where $\tau$ is the mean time between collisions.

\section*{Current and Current Density}
If $Q$ is the total amount of charge that has moved past a point in the wire, we define the current
$I$ in the wire to be the rate of charge flow
\[I\equiv \frac{dQ}{dt} A=\text{ampere}=C/s\]
 
For a steady current, the current $I$ during the time interval $\Delta t$ is
\[I=\frac{Q}{\Delta t}=\frac{eN_e}{\Delta t}=ei_e\]

The current density $J$ in a wire is the current per square meter of cross section
\[J=\text{current density}\equiv \frac{I}{A} = n_e e v_d\]
\subsection*{Key Points}
\begin{itemize}
    \item The current $I$ is opposite of the direction of motion of the electrons in a metal.
    \item The current into a junction between wires must equal the current out of it.
\end{itemize}

\pagebreak

\section*{Conductivity and Resistivity}
$J=n_e e v_d$ and $v_d = e\tau E / m$, resulting in the current density being
\[J=\frac{n_e e^2 \tau}{m}E\]

Since $n_e e^2 \tau / m$ depends only on the conducting material, the conductivity $\sigma$ of a
material is
\[\sigma = \frac{n_e e^2 \tau}{m}\]

Using conductivity,\
\[J=\sigma E\]

\begin{center}
\begin{tabular}{|c c c|} 
    \hline
    Metal & Resistivity ($\Omega m$) & Conductivity ($\Omega^{-1} m^{-1}$) \\ [0.5ex]
    \hline
    Aluminum & $2.8\times 10^{-8}$ & $3.5\times 10^7$ \\
    \hline
    Iron & $9.7\times 10^{-8}$ & $1.0\times 10^7$ \\
    \hline
    Copper & $1.7 \times 10^{-8}$ & $6.0\times 10^7$ \\
    \hline
    Gold & $2.4\times 10^{-8}$ & $4.1\times 10^7$ \\
    \hline
    Silver & $1.6\times 10^{-8}$ & $6.2\times 10^7$ \\
    \hline
\end{tabular}
\end{center}
where the resistivity $\rho = 1/\sigma$ is the inverse of the conductivity.

\vspace{1em}

Metals become better conductors at lower temperatures since they have higher conductivity and lower
resistivity. The complete loss of resistance at low temperatures is called superconductivity.

\section*{Resistance and Ohm's Law}
Since $E=\Delta V / L$,
\[I=\frac{A}{\rho L}\Delta V\] showing that the current is directly proportional to the potential
difference between the ends of a conductor.

\vspace{1em}

The resistance is $r=\rho L / A$, which results in current the current through a conductor being
\[I=\frac{\Delta V}{R}\quad\text{(Ohm's Law)}\]

\begin{enumerate}
    \item The potential difference $\Delta V_\text{wire}$ from the battery causes an electric field
    $E=\Delta V_\text{wire} L$ in the wire.
    \item The electric field establishes a current $I=JA=\sigma AE$ in the wire.
    \item The magnitude of the current is determined jointly by the battery and the wire's
    resistance to be $I=\Delta V_\text{wire}/R$.
\end{enumerate}

\subsection*{Resistors in Series}

\[R_\text{eq}=R_1 + R_2 + C_3 + \cdots\quad (\text{sequential resistors})\]
\[\Delta V_\text{eq}=\Delta V_1 + \Delta V_2 + \Delta V_3 + \cdots\quad (\text{sequential resistors})\]

\subsection*{Resistors in Parallel}

\[\frac{1}{R_\text{eq}}=\frac{1}{R_1} + \frac{1}{R_2} + \frac{1}{R_3} + \cdots\quad (\text{parallel resistors})\]

\section*{Wattage}
\[P=V\times I=I^2 \times R\] where $P$ is the power in watts, $V$ is the voltage, $I$ is the current, and $R$ is the resistance.

\subsection*{Temperature Coefficient of Resistance}
\[R=R_0\times (1+\alpha \times (T-T_0))\]
where $R$ is the resistance at temperature $T$, $R_0$ is the resistance at temperature $T_0$, and $\alpha$ is the temperature coefficient of resistance

\end{document}