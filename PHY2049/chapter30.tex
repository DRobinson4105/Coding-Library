\documentclass{article}
\usepackage{amsmath}
\usepackage{bigints}
\usepackage{amssymb}

\title{Chapter 30 Electromagnetic Induction}
\author{David Robinson}
\date{}
\setlength{\parindent}{0pt}

\begin{document}
\maketitle

\section*{Motional emf}
The magnetic force on the charge carriers in a moving conductor creates an electric field $E=vB$
inside the conductor. The charge separation also creates an electric potential difference between
the two ends of the moving conductor.
\[\Delta V=V_\text{top} - V_\text{bottom} = -\int_0^l E_y dy =-\int_0^l (-vB)dy=vlB\]
So, the motion of the wire through a magentic field induces a potential difference $vlB$ between
the ends of the conductor.

\vspace{1em}
The positive charges are 90 degrees in the counter-clockwise direction from the velocity, while the
negative charges are 90 degrees in the clockwise direction from the velocity.

\vspace{1em}
The \textbf{motional emf} of a conductor moving with velocity $\vec{v}$ perpendicular to a magnetic
field $\vec{B}$ is \[\varepsilon=vlB\]

A current is induced in the circuit as the wire moves through a magnetic field. The current in the
circuit is an induced current. If the total resistance of the circuit is $R$, the induced current
is \[I=\frac{\varepsilon}{R}=\frac{vlB}{R}\]

The magnitude of the magnetic force on a current-carrying wire is
\[F_\text{mag}=F_\text{pull}=IlB=\Big(\frac{vlB}{R}\Big)lB=\frac{vl^2B^2}{R}\]

The power provided to the circuit by pulling on the wire is
\[P_\text{input}=F_\text{pull} \cdot v=\frac{v^2 l^2 B^2}{R}\]

but the circuit also dissipates energy by transforming electric energy into the thermal energy.
\[P_\text{dissipated}=I^2 R=\frac{v^2 l^2 B^2}{R}\]

The rate at which work is done on the circuit exactly balances the rate at which energy is
dissipated.

\subsection*{Eddy Currents}
Eddy currents are induced when a metal sheet is pulled through a magnetic field. The magnetic force
on the eddy currents is oppose to the direction of $\vec{v}$.

\section*{Magnetic Flux}
\[A_\text{eff}=ab\cos\theta = A \cos\theta\]
\[\Phi_m = \vec{A}\cdot\vec{B}\]

In a nonuniform magnetic field,
\[\Phi_m = \int \vec{B}\cdot d \vec{A}\]

\section*{Lenz's Law}
There is an induced current in a closed, conducting loop if and only if the magnetic flux through
the loop is changing. The direction of the induced current is such that the induced magnetic field
opposes the change in the flux.

\begin{enumerate}
    \item Determine the direction of the applied magnetic field.
    \item Determine how the flux is changing.
    \item Determine the direction of an induced magnetic field that will oppose the change in flux.
    \begin{itemize}
        \item Increasing flux: The induced magnetic field points opposite the applied magnetic
        field.
        \item Decreasing flux: The induced magnetic field points in the same direction as the
        applied magnetic field.
        \item Steady flux: There is no induced magnetic field.
    \end{itemize}
\end{enumerate}

\section*{Faraday's Law}
An emf $\varepsilon$ is induced around a closed loop if the magnetic flux through the loop changes.
The magnitude of the emf is
\[\varepsilon=\Big|\frac{d\Phi_m}{dt}\Big|\]
and the direction of the emf is such as to drive an induced current in the direction given by
Lenz's law.

\vspace{1em}
The induced emf is the emf associated with a changing magnetic flux.
\[I_\text{induced} = \frac{\varepsilon}{R}\]

\section*{Induced Fields}
An induced electric field is present whether there's a conducting loop or not.

\vspace*{0.5em}
The strength of the induced electric field inside a solenoid is
\[E_\text{inside}=\frac{r}{2}\Big|\frac{dB}{dt}\Big|\]

\section*{Induced Current Applications}
\begin{enumerate}
    \item Transformers
    \[\frac{V_2}{V_1}=\frac{N_2}{N_1}\]
\end{enumerate}
\section*{Inductors}
A coil is called an inductor because the potential difference across an inductor is an induced emf.

The inductance $L$ of a coil is the flux-to-current ratio:
\[L=\frac{\Phi_m}{I}\] with a unit of \textbf{henry},
$1 \text{ henry} = 1\: H \equiv 1\: Wb\: /\: A = 1\: T\: m^2\: /\: A$.

\vspace{1em}
Inductors become important circuit elements when currents are changing.

\vspace{1em}
The potential difference across an inductor, measured along the direction of the current, is
\[\Delta V_L = -L\frac{dI}{dt}\]

The total energy stored in an inductor is
\[U_L = L\int_0^I I dI = \frac{1}{2}LI^2\]

\section*{LC Circuits}
An LC circuit is a circuit consisting of an inductor and a capacitor.

\section*{LR Circuits}
An LR circuit is a circuit consisting of an inductor, a resistor, and sometimes a battery.
\end{document}