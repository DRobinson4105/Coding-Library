\documentclass{article}

\title{Midterm Notes}
\author{David Robinson}
\date{}
\setlength{\parindent}{0pt}

\begin{document}
\maketitle

\section*{File systems}
\textbf{What makes the unix file system "hierarchical"?}

It organizes files and directories in a tree-like structure, where there is a root directory and each directory can contain subdirectories and files, forming a parent-child relationship.

\vspace{1em}
\textbf{What is the difference between absolute vs. relative paths?}

Absolute paths start from the root directory while relative paths start from the working directory.

\vspace{1em}
\textbf{How are parent directories referenced in the file system?}

Parent directories are referenced using \texttt{..} in the filepath.

\section*{Navigation}
\textbf{What is the working directory and how do you display it?}

The working directory is the directory that a program is currently in and can be displayed with the \texttt{pwd} command.

\vspace{1em}
\textbf{What is the unix standard command to rename a file?}

The unix command to rename a file is \texttt{mv old-file-name new-file-name}.

\vspace{1em}
\textbf{What is tab-completion?}

Tab completion is a feature that automatically completes commands and file/directory names when Tab is pressed.
\end{document}