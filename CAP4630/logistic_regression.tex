% chktex-file 44

\documentclass{article}
\usepackage{amsmath}
\usepackage{amssymb}
\usepackage{tabularx}
\usepackage[a4paper, top=0.75in, bottom=0.75in]{geometry}

\title{Linear Regression}
\author{David Robinson}
\date{}
\setlength{\parindent}{0pt}

\begin{document}

\maketitle

\textbf{Logistic regression} is a supervised learning algorithm that predicts the probability of a binary outcome based on input features.

\subsection*{Sigmoid Function}

The \textbf{sigmoid function} transforms input values into output values that lie between 0 and 1.
\[\sigma(z)=\frac{1}{1+e^{-z}}\]

\subsection*{Cost Function}

In linear regression, mean squared error is used but will give results with local minima, which will not help with non-linear tasks, like logistic regression. Logistic regression uses log loss which is derived from the maxmimum likelihood estimation method.
\[\text{Cost}(h_\theta(x),y)=-\log(1-y+(2y-1)h_\theta(x))\]

Loss over a batch size of $m$,
\[J(\theta)=\frac{1}{m}\sum_{i=1}^m \text{Cost}(h_\theta(x),y)\]


At each epoch,
\[\theta_j := \theta_j-\alpha\frac{\partial}{\partial\theta_j}J(\theta)\]
\vspace{1em}
\centering
\renewcommand{\arraystretch}{1.5}
\begin{tabularx}{\textwidth}{|X|X|}
    \hline
    \textbf{Linear Regression} & \textbf{Logistic Regression} \\
    \hline
    Predicts a \textbf{continuous} dependent variable using independent variables. & Predicts a \textbf{categorical} (binary) dependent variable using independent variables. \\
    \hline
    Solves \textbf{regression} problems. & Solves \textbf{classification} problems. \\
    \hline
    Uses the \textbf{Least Squares} estimation method for accuracy. & Uses the \textbf{Maximum Likelihood} estimation method for accuracy. \\
    \hline
    Fits a \textbf{straight line} (best-fit line) to predict the output. & Fits an \textbf{S-curve} (logistic function) to classify the samples. \\
    \hline
    Requires a \textbf{linear relationship} between the dependent and independent variables. & Does \textbf{not} require a linear relationship between the dependent and independent variables. \\
    \hline
\end{tabularx}

\end{document}