\documentclass{article}
\usepackage{amsmath}
\usepackage{amssymb}
\usepackage{txfonts}

\title{Chapter 6 Orthogonality and Least Squares}
\author{David Robinson}
\date{}
\setlength{\parindent}{0pt}

\begin{document}
\maketitle

\section*{Inner Product, Length, and Orthogonality}

\subsubsection*{Theorem 1}
Let $\mathbf{u}$, $\mathbf{v}$, and $\mathbf{w}$ be vectors in $\mathbb{R}^n$, and let $c$ be a
scalar. Then
\begin{enumerate}
    \item $\mathbf{u}\cdot\mathbf{v}=\mathbf{v}\cdot\mathbf{u}$
    \item $(\mathbf{u}+\mathbf{v})\cdot\mathbf{w}=\mathbf{u}\cdot\mathbf{w}+\mathbf{v}\cdot
    \mathbf{w}$
    \item $(c\mathbf{u})\cdot\mathbf{v}=c(\mathbf{u}\cdot\mathbf{v})=u\cdot(c\mathbf{v})$
    \item $\mathbf{u}\cdot\mathbf{u}\geq 0$, and $\mathbf{u}\cdot\mathbf{u}=0$ if and only if
    $\mathbf{u}=0$
\end{enumerate}

\subsubsection*{The Length of a Vector}
The length (or norm) of $\mathbf{v}$ is the nonnegative scalar $\|\mathbf{v}\|$ defined by
\[\|\mathbf{v}\|=\sqrt{\mathbf{v}\cdot\mathbf{v}}=\sqrt{v_1^2+v_2^2+\cdots+v_n^2}\text{,\quad and
\quad}\|\mathbf{v}\|^2=\mathbf{v}\cdot\mathbf{v}\]

\subsubsection*{Distance in $\mathbb{R}^n$}
For $\mathbf{u}$ and $\mathbf{v}$ in $\mathbb{R}^n$, the \textbf{distance between u and v}, written
as dist$(\mathbf{u}, \mathbf{v})$, is the length of the vector $\mathbf{u}-\mathbf{v}$.
\[\text{dist}(\mathbf{u}, \mathbf{v})=\|\:\mathbf{u}-\mathbf{v}\:\|\]

\subsubsection*{Orthogonal Vectors}
Two vectors $\mathbf{u}$ and $\mathbf{v}$ in $\mathbb{R}^n$ are \textbf{orthogonal} (to each other)
if $\mathbf{u}\cdot\mathbf{v}=0$.

\subsubsection*{Theorem 2 --- The Pythagorean Theorem}
Two vectors $\mathbf{u}$ and $\mathbf{v}$ are orthogonal if and only if
$\|\:\mathbf{u}+\mathbf{v}\:\|^2 = \|\mathbf{u}\|^2 + \|\mathbf{v}\|^2$.

\subsubsection*{Theorem 3}
Let $A$ be an $m\times n$ matrix. The orthogonal complement of the row space of $A$ is the null
space of $A$, and the orthogonal complement of the column space of $A$ is the null space of $A^T$
\[{(\text{Row }A)}^\perp = \text{Nul }A\quad\text{and}\quad{(\text{Col }A)}^\perp=\text{Nul }A^T\]

\subsection*{Key Points}
\begin{enumerate}
    \item $\mathbf{u}\cdot\mathbf{v}=\mathbf{u}^T\times \mathbf{v}$
    \item A unit vector in the direction of a vector can be determined by dividing that vector by
    its length.
    \item $\|c\mathbf{v}\|$ is not always equal to $c\|\mathbf{v}\|$. Since length is always
    positive, the value of $\|c\mathbf{v}\|$ is positive for all values of c. However,
    $c\|\mathbf{v}\|$ is negative if $c$ is negative.
    \item The determinant of an orthogonal matrix is $\pm 1$.
\end{enumerate}

\section*{Orthogonal Sets}

\subsubsection*{Theorem 4}
If $S=\{\mathbf{u}_1, \ldots, \mathbf{u}_p\}$ is an orthogonal set of nonzero vectors in
$\mathbb{R}^n$, then $S$ is linearly independent and hence is a basis for the subspace spanned by
$S$.

\subsubsection*{Orthogonal Basis}
An \textbf{orthogonal basis} for a subspace $W$ of $\mathbb{R}^n$ is a basis for $W$ that is also
an orthogonal set.

\subsubsection*{Theorem 5}
Let $\{\mathbf{u}_1, \ldots, \mathbf{u}_p\}$ be an orthogonal basis for a subspace $W$ of
$\mathbb{R}^n$. For each $\mathbf{y}$ in $W$, the weights in the linear combination are
\[\mathbf{y}=c_1\mathbf{u}_1 + \cdots + c_p\mathbf{u}_p\text{\quad given by \quad} c_j=
\frac{\mathbf{y}\cdot\mathbf{u}_j}{\mathbf{u}_j\cdot \mathbf{u}_j}\quad (j=1, \ldots, p)\]

\subsubsection*{Theorem 6}
An $m\times n$ matrix $U$ has orthonormal columns if and only if $U^T U=I$.

\subsubsection*{Theorem 7}
Let $U$ be an $m\times n$ matrix with orthonormal columns, and let $\mathbf{x}$ and $\mathbf{y}$ be
in $\mathbb{R}^n$. Then
\begin{enumerate}
    \item $\|U\mathbf{x}\|=\|\mathbf{x}\|$
    \item $(U\mathbf{x})\cdot(U\mathbf{y})=\mathbf{x}\cdot\mathbf{y}$
    \item $(U\mathbf{x})\cdot(U\mathbf{y})=0$ if and only if $\mathbf{x}\cdot\mathbf{y}=0$
\end{enumerate}

\subsection*{Key Points}
\begin{enumerate}
    \item A set of vectors is orthogonal if each pair of distinct vectors from the set is
    orthogonal.
    \item The vector $\mathbf{\hat{y}}$ is the orthogonal projection of $\mathbf{y}$ onto
    $\mathbf{u}$.
    \[\mathbf{\hat{y}}=\Bigg(\frac{\mathbf{y\cdot u}}{\mathbf{u\cdot u}}\Bigg)\mathbf{u}\]
    \item $\mathbf{y}$ can be writtena s the sum of a vector in $\text{Span}\{\mathbf{u}\}$ and a
    vector orthogonal to $\mathbf{u}$.
    \[\mathbf{y}=\mathbf{\hat{y}}+\mathbf{z}\]
    \item An orthonormal set is an orthogonal set where all of the vectors are unit vectors.
    \item If $A$ is a matrix with orthonormal columns, then $\|A\mathbf{x}\|=\|\mathbf{x}\|$.
    \item If $U$ is an orthogonal matrix, $U^T=U^{-1}$.
    \item A matrix with orthogonal columns is an orthonormal matrix if the matrix is also square.
    \item If $U$ is a $m\times n$ matrix with orthonormal columns, $\mathbf{x}$ is in
    $\mathbb{R}^n$, $\|U\mathbf{x}\|={(U\mathbf{x})}^T(U\mathbf{x})$
\end{enumerate}

\section*{Orthogonal Projections}

\subsubsection*{Theorem 8 --- The Orthogonal Decomposition Theorem}

Let $W$ be a subspace of $\mathbb{R}^n$. Then each $\mathbf{y}$ in $\mathbb{R}^n$ can be written
uniquely in the form
\[\mathbf{y}=\mathbf{\hat{y}}+\mathbf{z}\]
where $\mathbf{\hat{y}}$ is in $W$ and $\mathbf{z}$ is in $W^\perp$. In fact, if
$\{\mathbf{u}_1, \ldots, \mathbf{u}_p\}$ is any orthogonal basis of $W$, then
\[\mathbf{\hat{y}}=\frac{\mathbf{y}\cdot\mathbf{u}_1}{\mathbf{u}_1\cdot\mathbf{u}_1}\mathbf{u}_1+
\cdots+\frac{\mathbf{y}\cdot\mathbf{u}_p}{\mathbf{u}_p\cdot\mathbf{u}_p}\mathbf{u}_p\]
and $\mathbf{z}=\mathbf{y}-\mathbf{\hat{y}}$

\subsubsection*{Theorem 9 --- The Best Approximation Theorem}

Let $W$ be a subspace of $\mathbb{R}^n$, let $\mathbf{y}$ be any vector in $\mathbb{R}^n$, and let
$\mathbf{\hat{y}}$ be the orthogonal projection of $\mathbf{y}$ onto $W$. Then $\mathbf{\hat{y}}$
is the closest point in $W$ to $\mathbf{y}$, in the sense that
\[\|\mathbf{y}-\mathbf{\hat{y}}\|<\|\mathbf{y}-\mathbf{v}\|\]
for all $\mathbf{v}$ in $W$ distinct from $\mathbf{\hat{y}}$.

\subsubsection*{Theorem 10}
If $\{\mathbf{u}_1, \ldots, \mathbf{u}_p\}$ is an orthonormal basis for a subspace $W$ of
$\mathbb{R}^n$, then
\[\text{proj}_W \mathbf{y}=(\mathbf{y}\cdot\mathbf{u}_1)\mathbf{u}_1+(\mathbf{y}\cdot\mathbf{u}_2)
\mathbf{u}_2+\cdots +(\mathbf{y}\cdot\mathbf{u}_p)\mathbf{u}_p\]

If $U=\begin{bmatrix}\mathbf{u}_1 & \mathbf{u}_2 & \cdots & \mathbf{u}_p \end{bmatrix}$, then
\[\text{proj}_W\mathbf{y}=U U^T \mathbf{y}\quad\text{for all \textbf{y} in }\mathbb{R}^n\]

\subsection*{Key Points}
\begin{enumerate}
    \item If $\mathbf{u}_1$ and $\mathbf{u}_2$ are orthogonal but $\mathbf{u}_3$ is not orthogonal
    to $\mathbf{u}_1$ or $\mathbf{u}_2$, a nonzero vector $\mathbf{v}$ in $\mathbb{R}^3$ that is
    orthogonal to $\mathbf{u}_1$ and $\mathbf{u}_2$ can be constructed through
    $\mathbf{v}=\mathbf{u}_3-\mathbf{\hat{u}}_3$.
\end{enumerate}

\section*{The Gram-Schmidt Process}

\subsubsection*{Theorem 11 --- The Gram-Schmidt Process}
Given a basis $\{\mathbf{x}_1, \ldots, \mathbf{x}_p\}$ for a nonzero subspace $W$ of
$\mathbb{R}^n$, define
\[\begin{array}{l}
    \mathbf{v}_1 = \mathbf{x}_1 \\
    \mathbf{v}_2 = \mathbf{x}_2 - \frac{\mathbf{x}_2\cdot \mathbf{v}_1}{\mathbf{v}_1\cdot
    \mathbf{v}_1} \mathbf{v}_1 \\
    \mathbf{v}_3 = \mathbf{x}_3 - \frac{\mathbf{x}_3\cdot \mathbf{v}_1}{\mathbf{v}_1\cdot
    \mathbf{v}_1} \mathbf{v}_1 - \frac{\mathbf{x}_3\cdot \mathbf{v}_2}{\mathbf{v}_2\cdot
    \mathbf{v}_2} \mathbf{v}_2 \\
    \vdots \\
    \mathbf{v}_p = \mathbf{x}_p - \frac{\mathbf{x}_p\cdot \mathbf{v}_1}{\mathbf{v}_1\cdot
    \mathbf{v}_1} \mathbf{v}_1 - \frac{\mathbf{x}_p\cdot \mathbf{v}_2}{\mathbf{v}_2\cdot
    \mathbf{v}_2} \mathbf{v}_2 - \cdots \frac{\mathbf{x}_p\cdot \mathbf{v}_{p-1}}
    {\mathbf{v}_{p-1}\cdot \mathbf{v}_{p-1}} \mathbf{v}_{p-1} \\
\end{array}\]

Then $\{\mathbf{v}_1, \ldots, \mathbf{v}_p\}$ is an orthogonal basis for $W$. In addition
\[\text{Span}\{\mathbf{v}_1, \ldots, \mathbf{v}_k\} = \text{Span}\{\mathbf{x}_1, \ldots,
\mathbf{x}_k\}\quad \text{for } 1 \leq k \leq p\]

\subsubsection*{Theorem 12 --- The QR Factorization}
If $A$ is an $m\times n$ matrix with linearly independent columns, then $A$ can be factored as
$A=QR$, where $Q$ is an $m\times n$ matrix whose columns form an orthonormal basis for
$\text{Col } A$ and $R$ is an $n\times n$ upper triangular invertible matrix with positive entries
on its diagonal.

\pagebreak
\section*{Least-Squares Problems}

If $A$ is $m\times n$ and $\mathbf{b}$ is in $\mathbb{R}^m$, a \textbf{least-squares solution} of
$A\mathbf{x}=\mathbf{b}$ is an $\mathbf{\hat{x}}$ in $\mathbb{R}^n$ such that
\[\|\mathbf{b}-A\mathbf{\hat{x}}\|\leq \|\mathbf{b} - A\mathbf{x}\|\]
for all $\mathbf{x}$ in $\mathbf{R}^n$.

\subsubsection*{Theorem 13}
The set of least-squares solutions of $A\mathbf{x}=\mathbf{b}$ coincides with the nonempty set of
solutions of the normal equations $A^T A\mathbf{x}=A^T\mathbf{b}$.

\subsubsection*{Theorem 14}
Let $A$ be an $m\times n$ matrix. The following statements are logically equivalent:
\begin{enumerate}
    \item The equation $A\mathbf{x}=\mathbf{b}$ has a unique least-squares solution for each
    $\mathbf{b}$ in $\mathbb{R}^m$.
    \item The columns of $A$ are linearly independent.
    \item The matrix $A^T A$ is invertible.
\end{enumerate}

When these statements are true, the least-squares solution $\mathbf{\hat{x}}$ is given by
\[\mathbf{\hat{x}}={(A^T A)}^{-1} A^T \mathbf{b}\]

\subsubsection*{Theorem 15}
Given an $m\times n$ matrix $A$ with linearly independent columns, let $A=QR$ be a $QR$
factorization of $A$. Then, for each $\mathbf{b}$ in $\mathbb{R}^m$, the equation
$A\mathbf{x}=\mathbf{b}$ has a unique least-squares solution, given by
\[\mathbf{\hat{x}}=R^{-1}Q^T \mathbf{b}\]

\subsection*{Key Points}
\begin{enumerate}
    \item The least-squares error is the distance from $\mathbf{b}$ to $A\mathbf{\hat{x}}$,
    $\|\mathbf{b}-A\mathbf{\hat{x}}\|$.
    \item If $A$ is an orthogonal set, the weights that form the orthogonal projection of
    $\mathbf{b}$ onto $\text{Col }A$, $\frac{\mathbf{y}\cdot\mathbf{u}_j}{\mathbf{u}_j\cdot
    \mathbf{u}_j}$, form the least-squares solution of $A\mathbf{x}=\mathbf{b}$.
    \item If $\mathbf{b}$ is in the column space of $A$, then every solution of
    $A\mathbf{x}=\mathbf{b}$ is a least-squares solution.
    \item A least-squares solution of $A\mathbf{x}=\mathbf{b}$ is a list of weights that, when applied to the columns of $A$, produces the orthogonal projection of $\mathbf{b}$ onto Col $A$.
\end{enumerate}

\pagebreak

\section*{Machine Learning and Linear Models}

To find the equation $y=\beta_0 + \beta_1 X$ of the least-squares line that best fits data points
$(x_1, y_1), \ldots, (x_n, y_n)$, find the least-squares solution of $X\mathbf{\beta}=\mathbf{y}$,
where the design matrix $X$, the parameter vector $\mathbf{\beta}$, and observation vector
$\mathbf{y}$ are
\[X=\begin{bmatrix}
    1 & x_1 \\ 
    \vdots & \vdots \\
    1 & x_n
\end{bmatrix}\quad \mathbf{\beta} = \begin{bmatrix}
    \beta_0 \\
    \beta_1
\end{bmatrix}\quad \mathbf{y} = \begin{bmatrix}
    y_1 \\
    \vdots \\
    y_n
\end{bmatrix}\]
\[\mathbf{\beta}={(X^T X)}^{-1} X^T \mathbf{y}\]

\section*{Inner Product Spaces}

An \textbf{inner product} on a vector space $V$ is a function that, to each pair of vectors
$\mathbf{u}$ and $\mathbf{v}$ in $V$, associates a real number
$\langle\mathbf{u},\mathbf{v}\rangle$ and satisfies the following axioms, for all $\mathbf{u}$,
$\mathbf{v}$, and $\mathbf{w}$ in $V$ and all scalars $c$:
\begin{enumerate}
    \item $\langle\mathbf{u},\mathbf{v}\rangle=\langle\mathbf{v},\mathbf{u}\rangle$
    \item $\langle\mathbf{u+v},\mathbf{w}\rangle=\langle\mathbf{u},\mathbf{w}\rangle+\langle
    \mathbf{v},\mathbf{w}\rangle$
    \item $\langle c\mathbf{u},\mathbf{v}\rangle=c\langle\mathbf{u},\mathbf{v}\rangle$
    \item $\langle\mathbf{u},\mathbf{u}\rangle\geq 0\quad\text{and}\quad\langle\mathbf{u},
    \mathbf{u}\rangle=0\text{ if and only if }\mathbf{u=0}$
\end{enumerate}

A vector space with an inner product is called an \textbf{inner product space}.

\subsubsection*{Theorem 16 --- The Cauchy-Shwarz Inequality}
For all $\mathbf{u}, \mathbf{v}$ in $V$,
\[|\langle\mathbf{u},\mathbf{v}\rangle |\leq\|\mathbf{u}\|\:\|\mathbf{v}\|\]

\subsubsection*{Theorem 17 --- The Triangle Inequality}
For all $\mathbf{u}, \mathbf{v}$ in $V$,
\[\|\mathbf{u} + \mathbf{v}\| \leq \|\mathbf{u}\| + \|\mathbf{v}\|\]

\subsection*{Key Points}
\begin{enumerate}
    \item $\langle\mathbf{p},\mathbf{q}\rangle=\mathbf{p}(t_1)\mathbf{q}(t_1) + \cdots +
    \mathbf{p}(t_n)\mathbf{q}(t_n)$
    \item The orthogonal projection of a polynomial in an inner product space is 
    \[\mathbf{\hat{q}}=\frac{\langle\mathbf{q},\mathbf{p_1}\rangle}{\langle\mathbf{p_1},
    \mathbf{p_1}\rangle} \mathbf{p_1} + \cdots + \frac{\langle\mathbf{q},\mathbf{p_n}\rangle}
    {\langle\mathbf{p_n}, \mathbf{p_n}\rangle} \mathbf{p_n}\]
\end{enumerate}

\section*{Applications of Inner Product Spaces}

Let $C\:[0,2\pi]$ be a space with the inner product
$\langle f, g \rangle = \int_0^{2\pi} f(t)g(t)dt$. The function which approximates as closely as
desired for any function in $C\:[0,2\pi]$ is 
\[\frac{a_0}{2} + a_1\cos t + \cdots a_n \cos nt + b_1 \sin t + \cdots + b_n \sin nt\]

where $\frac{a_0}{2} = \frac{1}{2\pi}\int_0^{2\pi} f(t) \: dt$,
$a_k=\frac{1}{\pi}\int_0^{2\pi}f(t)\cos kt \: dt$, and
$b_k=\frac{1}{\pi}\int_0^{2\pi}f(t)\sin kt \: dt$
\end{document}