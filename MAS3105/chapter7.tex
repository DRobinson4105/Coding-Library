\documentclass{article}
\usepackage{amsmath}
\usepackage{amssymb}
\usepackage{txfonts}

\title{Chapter 6 Symmetric Matrices and Quadratic Forms}
\author{David Robinson}
\date{}
\setlength{\parindent}{0pt}

\begin{document}
\maketitle

\section*{Diagonalization of Symmetric Matrices}

A \textbf{symmetric} matrix is a matrix $A$ such that $A^T = A$.

\subsubsection*{Theorem 1}
If $A$ is symmetric, then any two eigenvectors from different eigenspaces are orthogonal.

\subsubsection*{Theorem 2}
An $n\times n$ matrix $A$ is orthogonally diagonalizable if and only if $A$ is a symmetric matrix.

\subsubsection*{Theorem 3 --- The Spectral Theorem for Symmetric Matrices}
An $n\times n$ symmetric matrix $A$ has the following properties:
\begin{enumerate}
    \item $A$ has $n$ real eigenvalues, counting multiplicities.
    \item The dimension of the eigenspace for each eigenvalue $\lambda$ equals the multiplicity of
    $\lambda$ as a root of the characteristic equation.
    \item The eigenspaces are mutually orthogonal, in the sense that eigenvectors corresponding to
    different eigenvalues are orthogonal.
    \item $A$ is orthogonally diagonalizable.
\end{enumerate}

\subsubsection*{Spectral Decomposition}
\[A=\lambda_1\mathbf{u}_1\mathbf{u}_1^T+\lambda_2\mathbf{u}_2\mathbf{u}_2^T+\cdots +\lambda_n
\mathbf{u}_n\mathbf{u}_n^T\]

\subsection*{Key Points}
\begin{enumerate}
    \item A matrix $U$ is orthogonal if $U^T U = I$, and if so, $U^T=U^{-1}$.
    \item A matrix $A$ can be orthogonally diagonalized by finding the $n$ eigenvalues and forming
    $D$ as a diagonal matrix of the eigenvalues and $P$ as the normalized orthogonal eigenvectors
    for the eigenvalues. (Use Gram-Schmidt Process to form orthogonal basis from eigenvectors).
    \item Multiplying a column vector $u$ of $\mathbb{R}^n$ on the right by $u^T x$ is the same as
    multiplying the column vector by the scalar $u\cdot x$.
\end{enumerate}

\end{document}