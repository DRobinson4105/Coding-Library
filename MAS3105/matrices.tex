\documentclass{article}
\usepackage{amsmath}
\usepackage{amssymb}

\begin{document}
\setlength{\parindent}{0pt}

\section*{Linear Equations in Linear Algebra}

A matrix is in \textbf{echelon form} if:
\begin{enumerate}
    \item All nonzero rows are above any rows of all zeros.
    \item Each leading entry of a row is in a column to the right of the leading entry of the row
    above it.
    \item All entries in a column below a leading entry are zeros.
\end{enumerate}

\noindent
A matrix is in \textbf{reduced echelon form} if:
\begin{enumerate}
    \item It is in echelon form.
    \item The leading entry in each nonzero row is 1.
\end{enumerate}

\subsection*{Properties}
\begin{itemize}
    \item Two matrices are row equivalent if there exists a sequence of elementary row operations
    that transforms one matrix into the other.
    \item Each matrix is row equivalent to only one reduced echelon matrix.
    \item The echelon form of a matrix is not unique, but the reduced echelon form is unique.
\end{itemize}

\subsection*{Existence and Uniqueness Theorem}
A linear system is consistent if the rightmost column of echelon form of the augmented matrix is
not a pivot column.

\subsection*{Row Reduction Algorithm}
The row reduction algorithm leads directly to an explicit description of the solution set of a
linear system when the algorithm is applied to the augmented matrix of the system, leading to a
general solution of a system.
\begin{enumerate}
    \item Forward Phase (reducing a matrix to echelon form)
    \begin{enumerate}
        \item Begin with the leftmost nonzero column. This is a pivot column. The pivot position is
        at the top.
        \item Select a nonzero entry in the pivot column as a pivot. If necessary, interchange rows
        to move this entry into the pivot position.
        \item Use row replacement operations to create zeros in all positions below the pivot.
        \item Ignore the row containing the pivot position and all rows above it.
        \item Repeat until there are no more nonzero rows to modify.
    \end{enumerate}
    \item Backward Phase (reducing a matrix to reduced echelon form)
    \begin{enumerate}
        \item Beginning with the rightmost pivot and working upward and to the left, create zeros
        above each pivot. If a pivot is not 1, make it 1 by a scaling operation.
    \end{enumerate}
\end{enumerate}

\subsection*{Span (Linear Combination)}
\begin{itemize}
    \item The span of two vectors, $\text{Span}\{\mathbf{v}_1, \mathbf{v}_2\}$, represents all
    vectors that can be reached by scaling and adding the two vectors.
    \item If the system consisting of vectors $\mathbf{v}_1$, $\mathbf{v}_2$, and $\mathbf{b}$ is
    consistent, then $\mathbf{b}$ is in $\text{Span}\{\mathbf{v}_1, \mathbf{v}_2\}$.
    \item A matrix can only span $\mathbb{R}^n$ if it has pivot positions in $n$ rows.
\end{itemize}

\subsection*{Matrix Equation $A\mathbf{x}=b$}
$A\mathbf{x}=b$ can be represented as a vector or matrix equation.
\[\begin{split}
    ax_1+bx_2+cx_3=d \\
    ex_1+fx_2+gx_3=h
\end{split}\]

Vector Equation:
\[x_1\begin{bmatrix} a \\ e \end{bmatrix} + x_2\begin{bmatrix} b \\ f \end{bmatrix} + x_3
\begin{bmatrix} c \\ h \end{bmatrix} = \begin{bmatrix} d \\ h \end{bmatrix}\]
Matrix Equation:
\[\begin{bmatrix} a & b & c \\ e & f & g \end{bmatrix}
\begin{bmatrix} x_1 \\ x_2 \\ x_3 \end{bmatrix} = \begin{bmatrix} d \\ h \end{bmatrix}\]
Augmented Matrix:
\[\begin{bmatrix}
    a & b & c & d \\
    e & f & g & h
\end{bmatrix}\]

Let $A$ be an $m \times n$ matrix. Then the following statements are logically equivalent. That is,
for a particular $A$, either they are all true statements or they are all false.
\begin{enumerate}
    \item For each $\textbf{b}$ in $\mathbb{R}^m$, the equation $A\mathbf{x}=\mathbf{b}$ has a
    solution.
    \item Each $\textbf{b}$ in $\mathbb{R}^m$ is a linear combination of the columns of $A$.
    \item The columns of $A$ span $\mathbb{R}^m$.
    \item $A$ has a pivot position in every row.
\end{enumerate}

\subsection*{Homogeneous Equation}
A homogeneous equation is a linear equation in the form $Ax=0$

\begin{enumerate}
    \item The homogeneous equation always has at least one solution (the trivial solution), where
    $\textbf{x}=\textbf{0}$.
    \item The homogeneous equation has a nontrivial solution if the equation has at least one free
    variable.
    \item If the matrix $\textbf{A}$ has more columns than rows ($n > m$), the system often has
    infinitely many solutions.
    \item If $\textbf{A}$ has $n$ pivot columns, the columns of $\textbf{A}$ are linearly
    independent, since every variable is a basic variable.
\end{enumerate}

\subsection*{Parametric Vector Equation}
The equation can be represented in parametric vector form if there is a free variable so that all
of the other variables are represented in terms of the free variable. For example, if $x_3$ is a
free variable in $\mathbb{R}^3$, 
$x=\begin{bmatrix} c \\ d \end{bmatrix} + x_3
\begin{bmatrix} a \\ b \\ 1\end{bmatrix}$, 
where $x_1=ax_3 + c$ and $x_2=bx_3 + d$.\\

If $\begin{bmatrix}
    x_1 \\ x_2 \\ x_3
\end{bmatrix} = \begin{bmatrix}
    a + bx_3 \\ c + dx_3 \\ e + fx_3
\end{bmatrix}$, the equation geometrically describes a line through $\begin{bmatrix}
    a \\ c \\ e
\end{bmatrix}$ parallel to $\begin{bmatrix}
    b \\ d \\ f
\end{bmatrix}$

The solution set of $A\mathbf{x}=\mathbf{b}$ is the set of all vectors of the form
$\mathbf{w}=\mathbf{p}+\mathbf{v}_h$, where $\mathbf{v}_h$ is any solution of the equation
$A\mathbf{x}=\mathbf{0}$, only when the equation $A\mathbf{x}=\mathbf{b}$ is consistent for some
given $\mathbf{b}$, and there exists a vector $\mathbf{p}$ such that $\mathbf{p}$ is a solution.

\subsection*{Linear Independence}
An indexed set of vectors $\{\mathbf{v}_1,\cdots,\mathbf{v}_p\}$ in $\mathbb{R}^n$ is said to be linearly independent if the vector equation $x_1\mathbf{v}_1+\cdots + x_p\mathbf{v}_p=0$ has only the trivial solution.
\begin{itemize}
    \item If a set $S=\{\mathbf{v}_1,\cdots, \mathbf{v}_p\}$ in $\mathbb{R}^n$ contains the zero vector, then the set is linearly dependent.
    \item Two vectors are linearly dependent if they live on a line through the origin.
\end{itemize}

\subsection*{Other}
\begin{itemize}
    \item Any list of five real numbers is a vector in $\mathbb{R}^n$
    \item $\{\mathbf{a}_1, \mathbf{a}_2, \mathbf{a}_3\}$ consists of the vectors, $\mathbf{a}_1$,
    $\mathbf{a}_2$, and $\mathbf{a}_3$
\end{itemize}

\end{document}
