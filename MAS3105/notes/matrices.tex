\documentclass{article}
\usepackage{amsmath}
\usepackage{amssymb}

\begin{document}
\setlength{\parindent}{0pt}
\setlength{\parskip}{1em}
\section*{Matrices}

A matrix is in \textbf{echelon form} if:
\begin{enumerate}
    \item All nonzero rows are above any rows of all zeros
    \item Each leading entry of a row is in a column to the right of the leading entry of the row
    above it
    \item All entries in a column below a leading entry are zeros
\end{enumerate}

\noindent
A matrix is in \textbf{reduced echelon form} if:
\begin{enumerate}
    \item It is in echelon form
    \item The leading entry in each nonzero row is 1
\end{enumerate}

\noindent
A linear system is consistent if and only if the rightmost column of the augmented matrix is not a
pivot column. That is, if and only if an echelon form of the augmented matrix has no row of the
form $\begin{bmatrix} 0 & \cdots & 0 & b \end{bmatrix}$ with $b$ being nonzero.


Let $A$ be an $m \times n$ matrix. Then the following statements are logically equivalent. That is, for a particular $A$, either they are all true statements or they are all false.
\begin{enumerate}
    \item For each $\textbf{b}$ in $\mathbb{R}^m$, the equation $A\mathbf{x}=\mathbf{b}$ has a solution
    \item Each $\textbf{b}$ in $\mathbb{R}^m$ is a linear combination of the columns of $A$
    \item The columns of $A$ span $\mathbb{R}^m$
    \item $A$ has a pivot position in every row
\end{enumerate}

\section*{Homogeneous Equation}
A linear equation in the form $Ax=0$ where:
\begin{itemize}
    \item $\textbf{A}$ is an $m\times n$ matrix
    \item $\textbf{x}$ is a vector in $\mathbb{R}^n$
    \item $\textbf{0}$ is the zero vector in $\mathbb{R}^m$
\end{itemize}

\subsection*{Properties}
\begin{enumerate}
    \item The homogeneous equation always has at least one solution (the trivial solution), where $\textbf{x}=\textbf{0}$
    \item If the matrix $\textbf{A}$ has more columns than rows ($n > m$), the system often has infinitely many solutions
    \item If $\textbf{A}$ has $n$ pivot columns, the columns of $\textbf{A}$ are linearly independent, since every variable is a basic variable
\end{enumerate}

If T: $\mathbb{R}^n\rightarrow\mathbb{R}^m$ is a linear transformation, then there exists a unique matrix $A$ such that the following equation is true.
\[T(\textbf{x})=A\textbf{x}\text{ for all } \textbf{x} \text{ in } \mathbb{R}^n\]
In fact, $A$ is the $m\times n$ matrix whose jth column is the vector $T(\textbf{e}_j)$ where $\textbf{e}_j$ is the jth column of the identity matrix in $\mathbb{R}^n$, as shown in the equation, $A=\begin{bmatrix}
    T(e_1) & \cdots & T(e_n)
\end{bmatrix}$

The mapping T: $\mathbb{R}^n\rightarrow\mathbb{R}^m$ is said to be one-to-one if each \textbf{b} in $\mathbb{R}^m$ is the image of at most one \textbf{x} in $\mathbb{R}^n$.

The mapping T: $\mathbb{R}^n\rightarrow\mathbb{R}^m$ is said to be onto $\mathbb{R}^m$ if each \textbf{b} in $\mathbb{R}^m$ is the image of at least one \textbf{x} in $\mathbb{R}^n$.

\end{document}