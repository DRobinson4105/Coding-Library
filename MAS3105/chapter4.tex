\documentclass{article}
\usepackage{amsmath}
\usepackage{amssymb}

\title{Chapter 4 Vector Spaces}
\author{David Robinson}
\date{}
\setlength{\parindent}{0pt}

\begin{document}
\maketitle

\section*{Vector Spaces and Subspaces}
A \textbf{vector space} is a nonempty set $V$ of objects, called vectors, on which are defined two
operations, called addition and mulitplication by scalars (real numbers), subject to the ten rules
listed below. The rules must hold for all vectors $\mathbf{u}$, $\mathbf{v}$, and $\mathbf{w}$ in
$V$ and for all scalars $c$ and $d$.

\begin{enumerate}
    \item The sum of $\mathbf{u}$ and $\mathbf{v}$, denoted by $\mathbf{u}+\mathbf{v}$, is in $V$.
    \item $\mathbf{u}+\mathbf{v}=\mathbf{v}+\mathbf{u}$.
    \item $(\mathbf{u}+\mathbf{v})+\mathbf{w}=\mathbf{u}+(\mathbf{v}+\mathbf{w})$.
    \item There is a \textbf{zero} vector $\mathbf{0}$ in $V$ such that
    $\mathbf{u}+\mathbf{0}=\mathbf{u}$.
    \item For each $\mathbf{u}$ in $V$, there is a vector $-\mathbf{u}$ in $V$ such that
    $\mathbf{u}+(-\mathbf{u})=\mathbf{0}$.
    \item The scalar multiple of $\mathbf{u}$ by $c$, denoted by $c\mathbf{u}$, is in $V$.
    \item $c(\mathbf{u}+\mathbf{v})=c\mathbf{u}+c\mathbf{v}$.
    \item $(c+d)\mathbf{u}=c\mathbf{u}+d\mathbf{u}$.
    \item $c(d\mathbf{u})=(cd)\mathbf{u}$.
    \item $1\mathbf{u}=\mathbf{u}$.
\end{enumerate}

A \textbf{subspace} of a vector space $V$ is a subset $H$ of $V$ that has three properties:
\begin{enumerate}
    \item The zero vector of $V$ is in $H$.
    \item $H$ is closed under vector addition. That is, for each $\mathbf{u}$ and $\mathbf{v}$ in
    $H$, the sum $\mathbf{u}+\mathbf{v}$ is in $H$.
    \item $H$ is closed under multiplication by scalars. That is, for each $\mathbf{u}$ in $H$ and
    each scalar $c$, the vector $c\mathbf{u}$ is in $H$.
\end{enumerate}

\subsubsection*{Theorem 1}
If $\mathbf{v}_1,\ldots, \mathbf{v}_p$ are in a vector space $V$, then
$\text{Span}\{\mathbf{v}_1,\ldots, \mathbf{v}_p\}$ is a subspace of $V$.

\subsection*{Key Points}
\begin{enumerate}
    \item If $\mathbf{v}$ is in $\mathbb{R}^3$, $H=\text{Span}\{\mathbf{v}\}$ is a subspace of
    $\mathbb{R}^3$.
\end{enumerate}

\pagebreak

\section*{Null Spaces, Column Spaces, Row Spaces, and Linear Transformations}
\subsection*{Null Spaces}
The \textbf{null space} of an $m\times n$ matrix $A$, written as $\text{Nul } A$, is the set of all
solutions of the homogeneous equation $A\mathbf{x}=\mathbf{0}$. In set notation,
\[\text{Nul } A = \{\mathbf{x} : \mathbf{x} \text{ is in } \mathbb{R}^n \text{ and }A\mathbf{x}=
\mathbf{0}\}\]

To find the vectors that span the null space:
\begin{enumerate}
    \item Row reduce the augmented matrix $\begin{bmatrix}A & 0\end{bmatrix}$ to reduced echelon
    form.
    \item Write the solution in terms of the free variables.
    \item The column vectors that are multiplied by the free variables in the previous step form
    the spanning set for $\text{Nul } A$.
\end{enumerate}

\subsubsection*{Theorem 2}
The null space of an $m\times n$ matrix $A$ is a subspace of $\mathbb{R}^n$. Equivalently, the set
of all solutions to a system $A\mathbf{x}=\mathbf{0}$ of $m$ homogeneous linear equations in $n$
unknowns is a subspace of $\mathbb{R}^n$.

\vspace{1em}

\subsection*{Column Spaces}
The \textbf{column space} of an $m\times n$ matrix $A$, written as $\text{Col } A$, is the set of
all linear combinations of the columns of $A$. If
$A=\begin{bmatrix}a_1 & \cdots & a_n\end{bmatrix}$, then
\[\text{Col } A = \text{Span}\{a_1, \ldots, a_n\}\]

To find the vectors that span the column space:
\begin{enumerate}
    \item Determine the pivot columns in the matrix.
    \item The pivot columns in the original matrix form the spanning set for $\text{Col } A$.
\end{enumerate}

\subsubsection*{Theorem 3}
The column space of an $m\times n$ matrix $A$ is a subspace of $\mathbb{R}^m$.

\subsection*{Row Spaces}
The \textbf{row space} of an $m\times n$ matrix $A$, written as $\text{Row } A$, is the set of all
linear combinations of the rows of $A$. If $A=\begin{bmatrix}r_1 \\ \cdots \\ r_n\end{bmatrix}$,
then
\[\text{Row } A = \text{Span}\{r_1, \ldots, r_n\}\]

To find the vectors that span the row space:
\begin{enumerate}
    \item Row reduce the matrix to echelon form.
    \item The pivot rows in the resulting matrix form the the spanning set for $\text{Row } A$.
\end{enumerate}

\subsection*{Linear Transformations}
A \textbf{linear transformation} $T$ from a vector space $V$ into a vector space $W$ is a rule that
assigns to each vector $\mathbf{x}$ in $V$ a unique vector $T(\mathbf{x})$ in $W$, such that
\begin{align*}
    (i) &\quad T(\mathbf{u} + \mathbf{v}) = T(\mathbf{u}) + T(\mathbf{v}) & \text{for all }
    \mathbf{u}, \mathbf{v} \text{ in } V \\
    (ii) &\quad T(c \mathbf{u}) = c T(\mathbf{u}) & \text{for all } \mathbf{u} \text{ in } V 
    text{ and all scalars } c.
\end{align*}

\section*{Linear Independent Sets; Bases}

\subsubsection*{Theorem 4}
An indexed set $\{v_1, \ldots, v_p\}$ of two or more vectors, with $v_1 \neq 0$, is linear
dependent if and only if some $v_j$ (with $j > 1$) is a linear combination of the preceding
vectors, $v_1, \ldots, v_{j-1}$.

\subsection*{Bases}
Let $H$ be a subspace of a vector space $V$. A set of vectors $\mathcal{B}$ in $V$ is a
\textbf{basis} for $H$ if
\begin{align*}
    (i) & \ \mathcal{B} \text{ is a linearly independent set, and} \\
    (ii) & \ \text{the subspace spanned by } \mathcal{B} \text{ coincides with } H;
    \text{ that is,}
\end{align*}
\[H = \text{Span } \mathcal{B}\]

If $B=\{\mathbf{b}_1, \ldots, \mathbf{b}_n\}$ is a basis for a vector space $V$
\begin{itemize}
    \item By the definition of a basis, $\mathbf{b}_1, \ldots, \mathbf{b}_n$ are in $V$.
    \item By the Unique Representation Theorem, for each $\mathbf{x}$ in $V$, there exists a unique set of scalars $c_1, \ldots, c_n$ such that $\mathbf{x}=c_1\mathbf{b}_1 + \cdots + c_n\mathbf{b}_n$.
    \item $\mathbf{b}_k=c_1\mathbf{b}_1 + \cdots + c_n\mathbf{b}_n$ for some unique set of scalars $c_1, \ldots, c_n$.
    \item Thus, the coordinate vector ${[\mathbf{b}_k]}_\mathcal{B}$ of $\mathbf{b}_k$ is $\mathbf{e}_k$, or the kth column of the $n\times n$ identity matrix.
\end{itemize}

\subsubsection*{Theorem 5 --- The Spanning Set Theorem}
Let $S=\{v_1, \ldots, v_p\}$ be a set in a vector space $V$, and let
$H=\text{Span}\{v_1, \ldots, v_p\}$.
\begin{enumerate}
    \item If one of the vectors in $S$, $v_k$, is a linear combination of the remaining vectors in
    $S$, then the set formed from $S$ by removing $v_k$ still spans $H$.
    \item If $H\neq {0}$, some subset of $S$ is a basis for $H$.
\end{enumerate}

\subsubsection*{Theorem 6}
The pivot columns of a matrix $A$ form a basis for $\text{Col } A$.

\subsubsection*{Theorem 7}
If two matrices $A$ and $B$ are row equivalent, then their row spacesare the same. If $B$ is in
echelon form, the nonzero rows of $B$ form a basis for the row space of $A$ as well as for that of
$B$.

\section*{Coordinate Systems}
\subsubsection*{Theorem 8 --- The Unique Representation Theorem}
Let $\mathcal{B}=\{\mathbf{b}_1, \ldots, \mathbf{b}_n\}$ be a basis for a vector space $V$. Then
for each $\mathbf{x}$ in $V$, there exists a unique set of scalars $c_1, \ldots, c_n$ such that
\[\mathbf{x}=c_1\mathbf{b}_1 + \cdots + c_n\mathbf{b}_n\]

\section*{Coordinates}
Suppose $\mathcal{B}=\{\mathbf{b}_1, \ldots, \mathbf{b}_n\}$ is a basis for a vector space $V$ and
$\mathbf{x}$ is in $V$. The coordinates of x relative to the basis $\mathcal{B}$ (or the
$\mathcal{B}$-coordinates of x) are the weights $c_1, \dots, c_n$ such that
$\mathbf{x}=c_1\mathbf{b}_1 + \cdots + c_n\mathbf{b}_n$.

\subsubsection*{Theorem 9}
Let $\mathcal{B} = \{\mathbf{b}_1, \ldots, \mathbf{b}_n\}$ be a basis for a vector space $V$. Then
the coordinate mapping $\mathbf{x} \mapsto {\left[ \mathbf{x} \right]}_{\mathcal{B}}$ is a
one-to-one linear transformation from $V$ onto $\mathbb{R}^n$. 
\[\mathcal{B}{\left[ \mathbf{x} \right]}_{\mathcal{B}}=\mathbf{x}\]

\subsection*{Key Points}
\begin{itemize}
    \item If $B=\left[ \mathbf{b}_1, \ldots, \mathbf{b}_n\right]$, then the change-of-coordinates
    matrix from $B$ to the standard basis in $\mathbb{R}^2$ is
    $\begin{bmatrix} \mathbf{b}_1 & \cdots & \mathbf{b}_n\end{bmatrix}$.
    \item If a set of coordinate vectors of each polynomial has a pivot position in each row, by
    isomorphism between $\mathbb{R}^3$ and $\mathbb{P}_2$, the set of polynomials spans
    $\mathbb{P}_2$. 
\end{itemize}

\section*{The Dimension of a Vector Space}
If a vector space $V$ is spanned by a finite set, then $V$ is said to be \textbf{finite-dimensional}, and the \textbf{dimension} of $V$, written as $\textbf{dim }V$, is the number of vectors in a basis for $V$. The dimension of the zero vector space $\{\mathbf{0}\}$ is defined to be zero. If $V$ is not spanned by a finite set, then $V$ is said to be \textbf{infinite-dimensional}.

\vspace{1em}

The \textbf{rank} of an $m\times n$ matrix $A$ is the dimension of the column space and the \textbf{nullity} of $A$ is the dimension of the null space.

\subsubsection*{Theorem 10}
If a vector space $V$ has a basis $\mathcal{B}=\{\mathbf{b}_1,\ldots, \mathbf{b}_n\}$, then any set in $V$ containing more than $n$ vectors must be linearly dependent.

\subsubsection*{Theorem 11}
If a vector space $V$ has a basis of $n$ vectors, then every basis of $V$ must consist of exactly $n$ vectors.

\subsubsection*{Theorem 12}
Let $H$ be a subspace of a finite-dimensional vector space $V$. Any linearly independent set in $H$ can be expanded, if necessary, to a basis for $H$. Also, $H$ is finite-dimensional and \[\text{dim }H\leq \text{dim }V\]

\subsubsection*{Theorem 13 --- The Basis Theorem}
Let $V$ be a $p$-dimensional vector space, $p\geq 1$. Any linearly independent set of exactly $p$ elements in $V$ is automatically a basis for $V$. Any set of exactly $p$ elements that spans $V$ is automatically a basis for $V$.

\subsubsection*{Theorem 14 --- The Rank Theorem}
The dimensions of the column space and the null space of an $m\times n$ matrix $A$ satisfy the equation
\[\text{rank }A+\text{nullity }A=\text{number of columns in }A\]

\subsubsection*{The Invertible Matrix Theorem}
Let $A$ be an $n\times n$ matrix. Then the following statements are each equivalent to the statement that $A$ is an invertible matrix.
\begin{enumerate}
    \item The columns of $A$ form a basis of $\mathbb{R}^n$.
    \item $\text{Col }A=\mathbb{R}^n$
    \item $\text{rank }A=n$
    \item $\text{nullity }A=0$
    \item $\text{Nul }A=\{\mathbf{0}\}$
\end{enumerate}

\subsection*{Dimensions}
The dimension of the:
\begin{itemize}
    \item $\text{Nul }A$ is the number of free variables in $A$.
    \item $\text{Col }A$ is the number of pivot columns in $A$.
    \item $\text{Row }A$ is the number of pivot rows in $A$.
\end{itemize}

\subsection*{Other Points}
\begin{itemize}
    \item If $\mathbb{P}_n$ is the space of all polynomials of degree at most $n$, its dimension is $n+1$.
\end{itemize}

\section*{Extra Notes}
\begin{enumerate}
    \item $\sin x \cos x = \frac{1}{2} \sin 2x$
\end{enumerate}
\end{document}